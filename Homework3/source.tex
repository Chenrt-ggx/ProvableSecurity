%% HOW TO USE THIS TEMPLATE:
%%
%% Ensure that you replace "YOUR NAME HERE" with your own name, in the
%% \studentname command below.  Also ensure that the "answers" option
%% appears within the square brackets of the \documentclass command,
%% otherwise latex will suppress your solutions when compiling.
%%
%% Type your solution to each problem part within
%% the \begin{solution} \end{solution} environment immediately
%% following it.  Use any of the macros or notation from the
%% header.tex that you need, or use your own (but try to stay
%% consistent with the notation used in the problem).
%%
%% If you have problems compiling this file, you may lack the
%% header.tex file (available on the course web page), or your system
%% may lack some LaTeX packages.  The "exam" package (required) is
%% available at:
%%
%% http://mirror.ctan.org/macros/latex/contrib/exam/exam.cls
%%
%% Other packages can be found at ctan.org, or you may just comment
%% them out (only the exam and ams* packages are absolutely required).

% the "answers" option causes the solutions to be printed
% \documentclass[11pt,addpoints]{exam}

%请使用xelatex编译
\documentclass[11pt,addpoints,answers]{exam}

% required macros -- get header.tex file from course web page
% !Mode:: "TeX:UTF-8"
% !TEX root=./hw1.tex
\usepackage{xeCJK}    % 写中文要用到
\usepackage{tikz}
\usepackage{amsmath,amsfonts,amssymb,amsthm}
\usepackage{fullpage}
\usepackage{times}
\usepackage{hyperref}
\usepackage{pdfsync}
\usepackage{microtype}
\usepackage{color}
\usepackage{cleveref}
\crefformat{footnote}{#2\footnotemark[#1]#3}
\definecolor{light-gray}{gray}{0.5}

\renewcommand{\tablename}{表}
\renewcommand{\figurename}{图}
\renewcommand{\proof}{证明}

% \theoremstyle{plain}
% \newtheorem{theorem}{定理~}
% \newtheorem{lemma}{引理~}
% \newtheorem{axiom}{公理~}
% \newtheorem{proposition}{命题~}
% \newtheorem{prop}{性质~}
% \newtheorem{corollary}{推论~}
% \newtheorem{conclusion}{结论~}
% \newtheorem{definition}{定义~}
% \newtheorem{conjecture}{猜想~}
% \newtheorem{example}{例~}
% \newtheorem{remark}{注~}
% \newtheorem{algorithm}{算法~}

%%% BLACKBOARD SYMBOLS

% \newcommand{\C}{\ensuremath{\mathbb{C}}}
\newcommand{\D}{\ensuremath{\mathbb{D}}}
\newcommand{\F}{\ensuremath{\mathbb{F}}}
% \newcommand{\G}{\ensuremath{\mathbb{G}}}
\newcommand{\J}{\ensuremath{\mathbb{J}}}
\newcommand{\N}{\ensuremath{\mathbb{N}}}
\newcommand{\Q}{\ensuremath{\mathbb{Q}}}
\newcommand{\R}{\ensuremath{\mathbb{R}}}
\newcommand{\T}{\ensuremath{\mathbb{T}}}
\newcommand{\Z}{\ensuremath{\mathbb{Z}}}
\newcommand{\QR}{\ensuremath{\mathbb{QR}}}

\newcommand{\Zt}{\ensuremath{\Z_t}}
\newcommand{\Zp}{\ensuremath{\Z_p}}
\newcommand{\Zq}{\ensuremath{\Z_q}}
\newcommand{\ZN}{\ensuremath{\Z_N}}
\newcommand{\Zps}{\ensuremath{\Z_p^*}}
\newcommand{\ZNs}{\ensuremath{\Z_N^*}}
\newcommand{\JN}{\ensuremath{\J_N}}
\newcommand{\QRN}{\ensuremath{\QR_{N}}}
\newcommand{\QRp}{\ensuremath{\QR_{p}}}

%%% THEOREM COMMANDS

% following are "theorem" style
\theoremstyle{plain}
\newtheorem{theorem}{定理}[section]
\newtheorem{lemma}[theorem]{引理}
\newtheorem{corollary}[theorem]{推论}
\newtheorem{proposition}[theorem]{命题}
\newtheorem{claim}[theorem]{声明}
\newtheorem{fact}[theorem]{事实}

% following are def style
\theoremstyle{definition}
\newtheorem{definition}[theorem]{定义}
\newtheorem{conjecture}[theorem]{推测}
\newtheorem{example}[theorem]{例}
\newtheorem{protocol}[theorem]{协议}

% following are remark style
\theoremstyle{remark}
\newtheorem{remark}[theorem]{注}
\newtheorem{note}[theorem]{注}
\newtheorem{exercise}[theorem]{练习}

% equation numbering style
\numberwithin{equation}{section}

%%% GENERAL COMPUTING

\newcommand{\bit}{\ensuremath{\set{0,1}}}
\newcommand{\pmone}{\ensuremath{\set{-1,1}}}

% asymptotics
\DeclareMathOperator{\poly}{poly}
\DeclareMathOperator{\polylog}{polylog}
\DeclareMathOperator{\negl}{negl}
\newcommand{\Otil}{\ensuremath{\tilde{O}}}

% probability/distribution stuff
\DeclareMathOperator*{\E}{E}
\DeclareMathOperator*{\Var}{Var}

% sets in calligraphic type
\newcommand{\calA}{\ensuremath{\mathcal{A}}}
\newcommand{\calD}{\ensuremath{\mathcal{D}}}
\newcommand{\calF}{\ensuremath{\mathcal{F}}}
\newcommand{\calH}{\ensuremath{\mathcal{H}}}
\newcommand{\calK}{\ensuremath{\mathcal{K}}}
\newcommand{\calM}{\ensuremath{\mathcal{M}}}
\newcommand{\calX}{\ensuremath{\mathcal{X}}}
\newcommand{\calY}{\ensuremath{\mathcal{Y}}}

% types of indistinguishability
\newcommand{\compind}{\ensuremath{\stackrel{c}{\approx}}}
\newcommand{\statind}{\ensuremath{\stackrel{s}{\approx}}}
\newcommand{\perfind}{\ensuremath{\equiv}}

% font for general-purpose algorithms
\newcommand{\algo}[1]{\ensuremath{\mathsf{#1}}}
% font for general-purpose computational problems
\newcommand{\problem}[1]{\ensuremath{\mathsf{#1}}}
% font for complexity classes
\newcommand{\class}[1]{\ensuremath{\mathsf{#1}}}

% complexity classes and languages
\renewcommand{\P}{\class{P}}
\newcommand{\BPP}{\class{BPP}}
\newcommand{\NP}{\class{NP}}
\newcommand{\coNP}{\class{coNP}}
\newcommand{\AM}{\class{AM}}
\newcommand{\coAM}{\class{coAM}}
\newcommand{\IP}{\class{IP}}

%%% "LEFT-RIGHT" PAIRS OF SYMBOLS

% inner product
\newcommand{\inner}[1]{\langle{#1}\rangle}
\newcommand{\innerfit}[1]{\left\langle{#1}\right\rangle}
% absolute value
\newcommand{\abs}[1]{\lvert{#1}\rvert}
\newcommand{\absfit}[1]{\left\lvert{#1}\right\rvert}
% a set
\newcommand{\set}[1]{\{{#1}\}}
\newcommand{\setfit}[1]{\left\{{#1}\right\}}
% parens
\newcommand{\parens}[1]{({#1})}
\newcommand{\parensfit}[1]{\left({#1}\right)}
% tuple = alias for parens
\newcommand{\tuple}[1]{\parens{#1}}
\newcommand{\tuplefit}[1]{\parensfit{#1}}
% square brackets
\newcommand{\bracks}[1]{[{#1}]}
\newcommand{\bracksfit}[1]{\left[{#1}\right]}
% rounding off
\newcommand{\round}[1]{\lfloor{#1}\rceil}
% floor function
\newcommand{\floor}[1]{\lfloor{#1}\rfloor}
% ceiling function
\newcommand{\ceil}[1]{\lceil{#1}\rceil}
% length of a string
\newcommand{\len}[1]{\lvert{#1}\rvert}
\newcommand{\lenfit}[1]{\left\lvert{#1}\right\rvert}
% length of some vector, element
\newcommand{\length}[1]{\lVert{#1}\rVert}
\newcommand{\lengthfit}[1]{\left\lVert{#1}\right\rVert}

%%% CRYPTO-RELATED NOTATION

% KEYS AND RELATED

\newcommand{\key}[1]{\ensuremath{#1}}

\newcommand{\pk}{\key{pk}}
\newcommand{\vk}{\key{vk}}
\newcommand{\sk}{\key{sk}}
\newcommand{\mpk}{\key{mpk}}
\newcommand{\msk}{\key{msk}}
\newcommand{\fk}{\key{fk}}
\newcommand{\id}{id}
\newcommand{\keyspace}{\ensuremath{\mathcal{K}}}
\newcommand{\msgspace}{\ensuremath{\mathcal{M}}}
\newcommand{\ctspace}{\ensuremath{\mathcal{C}}}
\newcommand{\tagspace}{\ensuremath{\mathcal{T}}}
\newcommand{\idspace}{\ensuremath{\mathcal{ID}}}
\newcommand{\concat}{\ensuremath{\|}}

% GAMES

% advantage
\newcommand{\advan}{\ensuremath{\mathbf{Adv}}}

% different attack models
\newcommand{\attack}[1]{\ensuremath{\text{#1}}}

\newcommand{\atk}{\attack{atk}}        % dummy attack
\newcommand{\indcpa}{\attack{ind-cpa}}
\newcommand{\indcca}{\attack{ind-cca}}
\newcommand{\anocpa}{\attack{ano-cpa}} % anonymous
\newcommand{\anocca}{\attack{ano-cca}}
\newcommand{\euacma}{\attack{eu-acma}} % forgery: adaptive chosen-message
\newcommand{\euscma}{\attack{eu-scma}} % forgery: static chosen-message
\newcommand{\suacma}{\attack{su-acma}} % strongly unforgeable

% ADVERSARIES

\newcommand{\attacker}[1]{\ensuremath{\mathcal{#1}}}

\newcommand{\Adv}{\attacker{A}}
\newcommand{\AdvA}{\attacker{A}}
\newcommand{\AdvB}{\attacker{B}}
\newcommand{\Dist}{\attacker{D}}
\newcommand{\Sim}{\attacker{S}}
\newcommand{\Ora}{\attacker{O}}
\newcommand{\Inv}{\attacker{I}}
\newcommand{\For}{\attacker{F}}

% CRYPTO SCHEMES

\newcommand{\scheme}[1]{\ensuremath{\text{#1}}}

% pseudorandom stuff
\newcommand{\prg}{\algo{PRG}}
\newcommand{\prf}{\algo{PRF}}
\newcommand{\prp}{\algo{PRP}}

% symmetric-key cryptosystem
\newcommand{\skc}{\scheme{SKC}}
\newcommand{\skcgen}{\algo{Gen}}
\newcommand{\skcenc}{\algo{Enc}}
\newcommand{\skcdec}{\algo{Dec}}

% public-key cryptosystem
\newcommand{\pkc}{\scheme{PKC}}
\newcommand{\pkcgen}{\algo{Gen}}
\newcommand{\pkcenc}{\algo{Enc}}
\newcommand{\pkcdec}{\algo{Dec}}

% digital signatures
\newcommand{\sig}{\scheme{SIG}}
\newcommand{\siggen}{\algo{Gen}}
\newcommand{\sigsign}{\algo{Sign}}
\newcommand{\sigver}{\algo{Ver}}

% message authentication code
\newcommand{\mac}{\scheme{MAC}}
\newcommand{\macgen}{\algo{Gen}}
\newcommand{\mactag}{\algo{Tag}}
\newcommand{\macver}{\algo{Ver}}

% key-encapsulation mechanism
\newcommand{\kem}{\scheme{KEM}}
\newcommand{\kemgen}{\algo{Gen}}
\newcommand{\kemenc}{\algo{Encaps}}
\newcommand{\kemdec}{\algo{Decaps}}

% identity-based encryption
\newcommand{\ibe}{\scheme{IBE}}
\newcommand{\ibesetup}{\algo{Setup}}
\newcommand{\ibeext}{\algo{Ext}}
\newcommand{\ibeenc}{\algo{Enc}}
\newcommand{\ibedec}{\algo{Dec}}

% hierarchical IBE (as key encapsulation)
\newcommand{\hibe}{\scheme{HIBE}}
\newcommand{\hibesetup}{\algo{Setup}}
\newcommand{\hibeext}{\algo{Extract}}
\newcommand{\hibeenc}{\algo{Encaps}}
\newcommand{\hibedec}{\algo{Decaps}}

% binary tree encryption (as key encapsulation)
\newcommand{\bte}{\scheme{BTE}}
\newcommand{\btesetup}{\algo{Setup}}
\newcommand{\bteext}{\algo{Extract}}
\newcommand{\bteenc}{\algo{Encaps}}
\newcommand{\btedec}{\algo{Decaps}}

% trapdoor functions
\newcommand{\tdf}{\scheme{TDF}}
\newcommand{\tdfgen}{\algo{Gen}}
\newcommand{\tdfeval}{\algo{Eval}}
\newcommand{\tdfinv}{\algo{Invert}}
\newcommand{\tdfver}{\algo{Ver}}

%%% PROTOCOLS

\newcommand{\out}{\text{out}}
\newcommand{\view}{\text{view}}

%%% COMMANDS FOR LECTURES/HOMEWORKS

\newcommand{\lecheader}{
\chead{
    \large \textbf{Lecture \lecturenum\\\lecturetopic}
}

\lhead{\small
    \textbf{\href{http://bhpan.buaa.edu.cn}{可证明安全技术}\\2022年秋季}
}

\rhead{\small
    \textbf{主讲: 张宗洋
}

\setlength{\headheight}{20pt}
\setlength{\headsep}{16pt}
}}

\newcommand{\hwheader}{
\chead{
    \Large \textbf{作业 \hwnum}
}

\lhead{\small
    \textbf{{可证明安全技术}\\2022年秋季}
}

\rhead{\small
    \textbf{主讲: 张宗洋\\姓名: \studentname}
}

\setlength{\headheight}{20pt}
\setlength{\headsep}{16pt}
\headrule
}

\newcommand{\examheader}{
\chead{
    \Large \textbf{测试 \\ \duedate}
}

\lhead{\small
    \textbf{\href{http://bhpan.buaa.edu.cn}{Introduction to modern cryptography}\\2022年秋季}}
    \rhead{\small \textbf{主讲: 张宗洋\\姓名: \studentname}}
    \setlength{\headheight}{20pt}
    \setlength{\headsep}{16pt}
    \headrule
}

\newcommand{\hint}[1]{\href{http://www.cims.nyu.edu/~regev/cgi-bin/hints/index.html?#1}{{\textcolor{light-gray}{I need a hint! (ID #1)}}}}

\newcommand{\hinttext}[2]{\href{http://www.cims.nyu.edu/~regev/cgi-bin/hints/index.html?#1}{{\textcolor{light-gray}{#2 (ID #1)}}}}

\newcommand{\hintcost}[2]{\href{http://www.cims.nyu.edu/~regev/cgi-bin/hints/index.html?#1}{{\textcolor{light-gray}{I need a hint for #2 points! (ID #1)}}}}

\newcommand{\moreinfo}[1]{\href{http://www.cims.nyu.edu/~regev/cgi-bin/hints/index.html?#1}{{\textcolor{light-gray}{I'm done solving and want to know more! (ID #1)}}}}


% VARIABLES

\newcommand{\hwnum}{3}
\newcommand{\studentname}{李田所-114514}

% END OF SUPPLIED VARIABLES

\hwheader                       % execute homework commands

\begin{document}

\pagestyle{head}                % put header on every page

% QUESTIONS START HERE.  PROVIDE SOLUTIONS WITHIN THE "solution"
% ENVIRONMENTS FOLLOWING EACH QUESTION.

\begin{questions}
    \question (3.9题) Consider a notion of indistinguishable encryption for multiple $distinct$ messages, i.e., where a scheme need not hide whether the same message is encrypted twice.

        \begin{enumerate}
            \item Modify Definition 3.18 to obtain a suitable definition of the above.
            \item Show that Construction 3.17 does not satisfy your definition.
            \item Give a construction of a $deterministic$ (stateless) encryption scheme that satisfies your definition.
        \end{enumerate}

        \begin{solution}
            %在此写入答案
        \end{solution}

    \question (3.11题) Let $F$ be a length-preseving pseudorandom function. For the following constructions of a keyed function $F':\{0,1\}^n\times\{0,1\}^{n-1}\rightarrow\{0,1\}^{2n}$, state whether $F'$ is a pseudorandom function. If yes, prove it; if not, show an attack.

        \begin{enumerate}
            \item $F_k'\overset{def}=F_k(0\parallel{x})\parallel{F_k}(0\parallel{x})$.
            \item $F_k'\overset{def}=F_k(0\parallel{x})\parallel{F_k}(1\parallel{x})$.
            \item $F_k'\overset{def}=F_k(0\parallel{x})\parallel{F_k}(x\parallel{0})$.
            \item $F_k'\overset{def}=F_k(0\parallel{x})\parallel{F_k}(x\parallel{1})$.
        \end{enumerate}

        \begin{solution}
            %在此写入答案
        \end{solution}

    \question (3.13题) Let $F$ be a keyed function and consider the following experiment:

        \indent The PRF indistinguishability experiment $PRF_{\AdvA,F}(n)$:

        \begin{enumerate}
            \item $A$ uniform bit $b\in\{0,1\}$ is chosen. If $b=1$ then choose uniform $k\in\{0,1\}^n$.
            \item $\AdvA$ is given $1^n$ for input. If $b=0$ then $\AdvA$ is given access to a uniform function $f\in\algo{Func}_n$. If $b=1$ then $\AdvA$ is instead given access to $F_k(\cdot)$.
            \item $\AdvA$ outputs a bit $b'$.
            \item The output of the experiment is defined to be $1$ if $b'=b$, and $0$ otherwise.
        \end{enumerate}

        Define pseudorandom functions using this experiment, and prove that your definition is equivalent to Definition 3.24.

        \begin{solution}
            %在此写入答案
        \end{solution}

    \question (3.16题) Prove that if $F$ is a length-preserving pseudorandom function, then $G(s)\overset{def}{=}F_s(\left\langle1\right\rangle)\parallel{F_s}(\left\langle2\right\rangle)\parallel\cdots\parallel{F_s}(\left\langle\ell\right\rangle)$, where $\left\langle{i}\right\rangle$ is the $n$-bit encoding of $i$, is a pseudorandom generator with expansion factor $\ell\cdot{n}$.

        \begin{solution}
            %在此写入答案
        \end{solution}

    \question (3.19题) Let $F$ be a pseudorandom permutation, and define a fixed-length ecryption scheme $(\textsf{Enc},\textsf{Dec})$ as follows: On input a key $k\in\{0,1\}^n$ and message $m\in\{0,1\}^{n/2}$, algorithm $\textsf{Enc}$ chooses a uniform string $r\in\{0,1\}^{n/2}$ and computes $c:=F_k(r\parallel{m})$.

        Show how to decrypt, and prove that this scheme is CPA-secure for messages of length $n/2$.

        \begin{solution}
            %在此写入答案
        \end{solution}

    \question(3.20题) Let $F$ be a length preserving pseudorandom function and $G$ be a pseudorandom generator with expansion factor $\ell(n)=n+1$. For each of the following encryption schemes, state whether the scheme has indistinguishable encryptions in the presence of an eavesdropper and whether it is CPA-secure.(In each case, the shared key is a uniform $k\in\{0,1\}^n$.) Explain your answer.

        \begin{enumerate}
            \item To encrypt $m\in\{0,1\}^{n+1}$, choose uniform $r\in\{0,1\}^n$ and output the ciphertext $\langle{r,G(r)\oplus{m}}\rangle$.
            \item To encrypt $m\in\{0,1\}^n$, output the ciphertext $m\oplus{F_k}(0^n)$.
            \item To encrypt $m\in\{0,1\}^{2n}$, parse $m$ as $\scriptstyle{m_1}\parallel{m_2}$ with $|m_1|=|m_2|$, then choose uniform $r\in\{0,1\}^n$ and send $\langle{r,m_1\oplus{F_k(r)},m_2\oplus{F_k(r+1)}}\rangle$.
        \end{enumerate}

        \begin{solution}
            %在此写入答案
        \end{solution}

    \question (3.25题) Let $F$ be a pseudorandom permutation. Consider the mode of operation in which a uniform value $IV\in\{0,1\}^n$ is chosen, and the $i$th ciphertext block $c_i$ is computed as $c_i:=F_k(IV+i+m_i)$, where addition is modulo $2^n$. Show that this scheme is not EAV-secure.

        \begin{solution}
            %在此写入答案
        \end{solution}

    \question ([选做]3.28题) For any function $g:\{0,1\}^n\leftarrow\{0,1\}^n$, define $g^\text{\$}(\cdot)$ to be a $probabilistic$ oracle that, on input $1^n$, chooses uniform $r\in\{0,1\}^n$ and returns $\left\langle{r,g(r)}\right\rangle$. A keyed function $F$ is a $weak\ pseudorandom\ function$ if for all PPT algorithms $D$, there exists a negligible function $\textsf{negl}$ such that:

        \begin{center}
            $\left|\operatorname{Pr}\left[D^{F_{k}^{\$}(\cdot)}\left(1^{n}\right)=1\right]-\operatorname{Pr}\left[D^{f^{\$}(\cdot)}\left(1^{n}\right)=1\right]\right|\leq\operatorname{negl}(n)$
        \end{center}

        where $k\in\{0,1\}^n$ and $f\in\textsf{Func}_n$ are chosen uniformly.

        \begin{enumerate}
            \item Prove that $F$ is pseudorandom then it is weakly pseudorandom.
            \item Let $F'$ be a pseudorandom function, and define:
                \begin{center}
                    $F_{k}(x)\stackrel{\text{def}}{=}\left\{\begin{array}{cl}F_{k}^{\prime}(x)&\text{if}x\text{is even}\\F_{k}^{\prime}(x+1)&\text{if}x\text{is odd}\end{array}\right.$
                \end{center}
                Prove that $F$ is weakly pseudorandom, but $not$ pseudorandom.
            \item Is CTR-mode encryption using a weak pseudorandom function necessarily CPA-secure? Prove your answer.
            \item Prove that Construction 3.28 is CPA-secure if $F$ is a weak pseudorandom function.
        \end{enumerate}

        \begin{solution}
            %在此写入答案
        \end{solution}

    \question (3.29题) What is the effect of a single bit flip in the ciphertext when using the CBC, OFB, and CTR modes of operation?

        \begin{solution}
            %在此写入答案
        \end{solution}

    \question (3.30题) What is the effect of a dropped ciphertext block (e.g., if the transmitted ciphertext $c_1,c_2,c_3\cdots$ is received as $c_1,c_3\cdots$) when using the CBC, OFB, and CTR modes of operation?

        \begin{solution}
            %在此写入答案
        \end{solution}

\end{questions}

\end{document}

%%% Local Variables:
%%% mode: latex
%%% TeX-master: t
%%% End:
