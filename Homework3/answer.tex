%% HOW TO USE THIS TEMPLATE:
%%
%% Ensure that you replace "YOUR NAME HERE" with your own name, in the
%% \studentname command below.  Also ensure that the "answers" option
%% appears within the square brackets of the \documentclass command,
%% otherwise latex will suppress your solutions when compiling.
%%
%% Type your solution to each problem part within
%% the \begin{solution} \end{solution} environment immediately
%% following it.  Use any of the macros or notation from the
%% header.tex that you need, or use your own (but try to stay
%% consistent with the notation used in the problem).
%%
%% If you have problems compiling this file, you may lack the
%% header.tex file (available on the course web page), or your system
%% may lack some LaTeX packages.  The "exam" package (required) is
%% available at:
%%
%% http://mirror.ctan.org/macros/latex/contrib/exam/exam.cls
%%
%% Other packages can be found at ctan.org, or you may just comment
%% them out (only the exam and ams* packages are absolutely required).

% the "answers" option causes the solutions to be printed
% \documentclass[11pt,addpoints]{exam}

%请使用xelatex编译
\documentclass[11pt,addpoints,answers]{exam}

% required macros -- get header.tex file from course web page
\input{header}

% VARIABLES

\newcommand{\hwnum}{3}
\newcommand{\studentname}{李田所-114514}

% END OF SUPPLIED VARIABLES

\hwheader                       % execute homework commands

\begin{document}

\pagestyle{head}                % put header on every page

% QUESTIONS START HERE.  PROVIDE SOLUTIONS WITHIN THE "solution"
% ENVIRONMENTS FOLLOWING EACH QUESTION.

\begin{questions}
    \question (3.9题) Consider a notion of indistinguishable encryption for multiple $distinct$ messages, i.e., where a scheme need not hide whether the same message is encrypted twice.

        \begin{enumerate}
            \item Modify Definition 3.18 to obtain a suitable definition of the above.
            \item Show that Construction 3.17 does not satisfy your definition.
            \item Give a construction of a $deterministic$ (stateless) encryption scheme that satisfies your definition.
        \end{enumerate}

        \begin{solution}
            \begin{enumerate}
                \item 只要调整实验 $\algo{PrivK}^{mult}_{\Adv,\Pi}(n)$ 的第一步,限制 $\vec{M_0}=(m_{0,1},...,m_{0,t})$ 中所有的 $m_{0,i}$ 互不相同,限制 $\vec{M_1}=(m_{1,1},...,m_{1,t})$ 中所有的 $m_{1,i}$ 互不相同(不要求所有的 $m_{0,i}$ 和所有的 $m_{1,i}$ 也互不相同)。
                \item 只要选择 $\vec{M_0}=(0^n,1\parallel0^{n-1})$ 和 $\vec{M_1}=(0^n,0^{n-1}\parallel1)$,对于结果 $\vec{C}=(c_0,c_1)$,如果 $c_0$ 和 $c_1$ 的最后一位相同,攻击者就可以确定 $b=0$(输出 $b'=0$);如果 $c_0$ 和 $c_1$ 的最后一位不同,攻击者就可以确定 $b=1$(输出 $b'=1$)。
                \item 由于无需考虑重复消息的问题,可以采用如下构造($F$ 为一个伪随机置换):
                    \begin{enumerate}
                        \item[*] $Gen$:对于输入 $1^n$,随机选择 $k\in\set{0,1}^n$,并输出 $k$。
                        \item[*] $Enc$:对于 $k\in\set{0,1}^n$ 和明文 $m\in\set{0,1}^n$,计算密文 $F_k(m)$ 并输出。
                        \item[*] $Dec$:对于 $k\in\set{0,1}^n$ 和密文 $c\in\set{0,1}^n$,计算明文 $F_k^{-1}(c)$ 并输出。
                    \end{enumerate}
            \end{enumerate}
        \end{solution}

    \question (3.11题) Let $F$ be a length-preseving pseudorandom function. For the following constructions of a keyed function $F':\{0,1\}^n\times\{0,1\}^{n-1}\rightarrow\{0,1\}^{2n}$, state whether $F'$ is a pseudorandom function. If yes, prove it; if not, show an attack.

        \begin{enumerate}
            \item $F_k'\overset{def}=F_k(0\parallel{x})\parallel{F_k}(0\parallel{x})$.
            \item $F_k'\overset{def}=F_k(0\parallel{x})\parallel{F_k}(1\parallel{x})$.
            \item $F_k'\overset{def}=F_k(0\parallel{x})\parallel{F_k}(x\parallel{0})$.
            \item $F_k'\overset{def}=F_k(0\parallel{x})\parallel{F_k}(x\parallel{1})$.
        \end{enumerate}

        \begin{solution}
            \begin{enumerate}
                \item $F'$ 不是伪随机函数,考虑 $x=0^{n-1}$,有 $F_k'(x)=F_k(0^n)\parallel{F_k}(0^n)$;注意到对于真随机的函数 $f$,$f(x)$ 的前半部分和后半部分相同的概率可忽略,$D$ 可根据输出的前半部分和后半部分是否相同(相同时输出 $1$)区分 $F_k'$ 和 $f$。
                \item $F'$ 是伪随机函数。下假设 $F'$ 不是伪随机函数(存在对 $F'$ 的高效识别器 $D'$),下面说明 $F$ 也不是伪随机函数,可按如下方式构造对 $F$ 的高效识别器 $D$:
                    \begin{enumerate}
                        \item[*] 启动对 $F'$ 识别器 $D'$。
                        \item[*] 对于 $D'$ 的每一次询问 $x$,$D$ 分别询问 $0\parallel{x}$ 和 $1\parallel{x}$ 的结果 $f(0\parallel{x})$ 和 $f(1\parallel{x})$,并向 $D'$ 返回 $f(0\parallel{x})\parallel{f}(1\parallel{x})$。
                        \item[*] 将 $D'$ 的输出作为 $D$ 的输出。
                    \end{enumerate}
                    由于 $F$ 是伪随机函数,假设不成立,$F'$ 是伪随机函数。
                \item $F'$ 不是伪随机函数,考虑 $x=0^{n-1}$,有 $F_k'(x)=F_k(0^n)\parallel{F_k}(0^n)$;注意到对于真随机的函数 $f$,$f(x)$ 的前半部分和后半部分相同的概率可忽略,$D$ 可根据输出的前半部分和后半部分是否相同(相同时输出 $1$)区分 $F_k'$ 和 $f$。
                \item $F'$ 不是伪随机函数,考虑 $x_1=0^{n-1}$ 和 $x_2=0^{n-2}\parallel1$,有 $F_k'(x_1)=F_k(0^n)\parallel{F_k}(0^{n-1}\parallel1)$ 和 $F_k'(x_2)=F_k(0^{n-1}\parallel1)\parallel{F_k}(0^{n-2}\parallel1^2)$;注意到对于真随机的函数 $f$,$f(x_1)$ 的后半部分和 $f(x_2)$ 的前半部分相同的概率可忽略,$D$ 可根据函数在 $x_1$ 处输出的后半部分和在 $x_2$ 处输出的前半部分是否相同(相同时输出 $1$)区分 $F_k'$ 和 $f$。
            \end{enumerate}
        \end{solution}

    \question (3.13题) Let $F$ be a keyed function and consider the following experiment:

        \indent The PRF indistinguishability experiment $PRF_{\AdvA,F}(n)$:

        \begin{enumerate}
            \item $A$ uniform bit $b\in\{0,1\}$ is chosen. If $b=1$ then choose uniform $k\in\{0,1\}^n$.
            \item $\AdvA$ is given $1^n$ for input. If $b=0$ then $\AdvA$ is given access to a uniform function $f\in\algo{Func}_n$. If $b=1$ then $\AdvA$ is instead given access to $F_k(\cdot)$.
            \item $\AdvA$ outputs a bit $b'$.
            \item The output of the experiment is defined to be $1$ if $b'=b$, and $0$ otherwise.
        \end{enumerate}

        Define pseudorandom functions using this experiment, and prove that your definition is equivalent to Definition 3.24.

        \begin{solution}
            \newline
            伪随机函数 $F$ 可通过上述实验定义为,对于任意多项式时间的算法 $\AdvA$,成立:
            \begin{equation}
                \begin{aligned}
                    \Pr[\algo{PRG}_{\AdvA,F}(n)=1]\leq\frac{1}{2}+negl(n)
                \end{aligned}
            \end{equation}
            等价性证明如下:
            \begin{equation}
                \begin{aligned}
                    \Pr[\algo{PRG}_{\AdvA,F}(n)=1]=&\Pr[b=0\wedge{b}^{'}=0]+\Pr[b=1\wedge{b}^{'}=1]\\
                    =&\frac{1}{2}Pr[D^{f(\cdot)}(1^n)=0]+\frac{1}{2}Pr[D^{F_k(\cdot)}(1^n)=1]\\
                    =&\frac{1}{2}\cdot(1-Pr[D^{f(\cdot)}(1^n)=1])+\frac{1}{2}Pr[D^{F_k(\cdot)}(1^n)=1]\\
                    =&\frac{1}{2}+\frac{1}{2}\cdot(Pr[D^{F_k(\cdot)}(1^n)=1]-Pr[D^{f(\cdot)}(1^n)=1])
                \end{aligned}
            \end{equation}
            从而可得如下等式:
            \begin{equation}
                \begin{aligned}
                    \Pr[\algo{PRG}_{\AdvA,F}(n)=1]-\frac{1}{2}=\frac{1}{2}\cdot(Pr[D^{F_k(\cdot)}(1^n)=1]-Pr[D^{f(\cdot)}(1^n)=1])\\
                    |\Pr[\algo{PRG}_{\AdvA,F}(n)=1]-\frac{1}{2}|=\frac{1}{2}\cdot|Pr[D^{F_k(\cdot)}(1^n)=1]-Pr[D^{f(\cdot)}(1^n)=1]|
                \end{aligned}
            \end{equation}
            若上述定义成立,则有:
            \begin{equation}
                \begin{aligned}
                    |\Pr[\algo{PRG}_{\AdvA,F}(n)=1]-\frac{1}{2}|=\frac{1}{2}\cdot|Pr[D^{F_k(\cdot)}(1^n)=1]-Pr[D^{f(\cdot)}(1^n)=1]|\leq{negl}(n)
                \end{aligned}
            \end{equation}
            从而可得:
            \begin{equation}
                \begin{aligned}
                    |Pr[D^{F_k(\cdot)}(1^n)=1]-Pr[D^{f(\cdot)}(1^n)=1]|\leq{negl}(n)
                \end{aligned}
            \end{equation}
            若定义 3.24 成立,则有:
            \begin{equation}
                \begin{aligned}
                    |\Pr[\algo{PRG}_{\AdvA,F}(n)=1]-\frac{1}{2}|=\frac{1}{2}\cdot|Pr[D^{F_k(\cdot)}(1^n)=1]-Pr[D^{f(\cdot)}(1^n)=1]|\leq{negl}(n)
                \end{aligned}
            \end{equation}
            从而可得:
            \begin{equation}
                \begin{aligned}
                    \frac{1}{2}-negl(n)\leq\Pr[\algo{PRG}_{\AdvA,F}(n)=1]\leq\frac{1}{2}+negl(n)
                \end{aligned}
            \end{equation}
        \end{solution}

    \question (3.16题) Prove that if $F$ is a length-preserving pseudorandom function, then $G(s)\overset{def}{=}F_s(\left\langle1\right\rangle)\parallel{F_s}(\left\langle2\right\rangle)\parallel\cdots\parallel{F_s}(\left\langle\ell\right\rangle)$, where $\left\langle{i}\right\rangle$ is the $n$-bit encoding of $i$, is a pseudorandom generator with expansion factor $\ell\cdot{n}$.

        \begin{solution}
            \newline
            假设 $G$ 不是伪随机生成器(存在对 $G$ 的高效识别器 $D'$),下面说明 $F$ 也不是伪随机函数,可按如下方式构造对 $F$ 的高效识别器 $D$:
                \begin{enumerate}
                    \item[*] 分别查询 $f(\left\langle1\right\rangle)$、$f(\left\langle2\right\rangle)$ 到 $f(\left\langle{l}\right\rangle)$。
                    \item[*] 拼接查询结果,并通过 $D'$ 进行区分:计算 $D'(f(\left\langle1\right\rangle)\parallel{f}(\left\langle2\right\rangle)\parallel...\parallel{f}(\left\langle{l}\right\rangle))$。
                    \item[*] 将 $D'$ 的输出作为 $D$ 的输出。
                \end{enumerate}
                由于 $F$ 是伪随机函数,假设不成立,$G$ 是伪随机生成器。
        \end{solution}

    \question (3.19题) Let $F$ be a pseudorandom permutation, and define a fixed-length ecryption scheme $(\textsf{Enc},\textsf{Dec})$ as follows: On input a key $k\in\{0,1\}^n$ and message $m\in\{0,1\}^{n/2}$, algorithm $\textsf{Enc}$ chooses a uniform string $r\in\{0,1\}^{n/2}$ and computes $c:=F_k(r\parallel{m})$.

        Show how to decrypt, and prove that this scheme is CPA-secure for messages of length $n/2$.

        \begin{solution}
            \newline
            解密方案显然为 $m:=F_k^{-1}(c)[n/2+1:n]$(将 $c$ 进行反置换后取后一半作为明文),下面证明此方案满足 CPA 安全。\\
            在此方案 $\Pi=(Gen,Enc,Dec)$ 的基础上构造方案 $\widetilde{\Pi}=(\widetilde{Gen},\widetilde{Enc},\widetilde{Dec})$,$\widetilde{\Pi}$ 除将 $\Pi$ 中使用的伪随机置换 $F$ 替换为真随机置换 $f$ 外与 $\Pi$ 保持一致。\\
            先证明以下等式成立:
            \begin{equation}
                \label{equal-1}
                \begin{aligned}
                    |\Pr[\algo{PrivK}^{cpa}_{\AdvA,\Pi}(n)=1]-\Pr[\algo{PrivK}^{cpa}_{\AdvA,\widetilde{\Pi}}(n)=1]|\leq{negl}(n)
                \end{aligned}
            \end{equation}
            假设上式不成立即存在攻击者 $\AdvA$ 可以高效区分 $\Pi$ 和 $\widetilde{\Pi}$,可按照如下方式构造可高效区分 $F$ 与 $f$ 的区分器 $D$:
            \begin{enumerate}
                \item 启动 $\AdvA(1^n)$,当 $\AdvA$ 向黑箱发出一次查询时,$D$ 随机选取 $r\in\set{0,1}^{n/2}$,查询 $F_k(r\parallel{m})$ 并返回给 $\AdvA$。
                \item 当 $\AdvA$ 给出消息 $m_0$ 和 $m_1$ 时,随机选择 $b\in\set{0,1}$,$D$ 随机选取 $r\in\set{0,1}^{n/2}$,查询 $F_k(r\parallel{m_b})$ 并返回给 $\AdvA$。
                \item 按第 1 步持续回答 $\AdvA$ 向黑箱发出查询,直到 $\AdvA$ 输出 $b'$,当 $b'=b$ 时 $D$ 输出 $1$,否则 $D$ 输出 $0$。
            \end{enumerate}
            注意到以下两式:
            \begin{equation}
                \begin{aligned}
                   \Pr[\algo{PrivK}^{cpa}_{\AdvA,\Pi}(n)=1]=Pr[D^{F_k(\cdot)}(1^n)=1]
                \end{aligned}
            \end{equation}
            \begin{equation}
                \begin{aligned}
                   \Pr[\algo{PrivK}^{cpa}_{\AdvA,\widetilde{\Pi}}(n)=1]=Pr[D^{f(\cdot)}(1^n)=1]
                \end{aligned}
            \end{equation}
            由于 $F$ 为伪随机函数,假设不成立,即不存在攻击者 $\AdvA$ 可以高效区分 $\Pi$ 和 $\widetilde{\Pi}$。设 $\AdvA$ 查询黑箱的次数存在多项式上限 $q(\cdot)$,再证明以下等式成立:
            \begin{equation}
                \label{equal-2}
                \begin{aligned}
                   \Pr[\algo{PrivK}^{cpa}_{\AdvA,\widetilde{\Pi}}(n)=1]\leq\frac{1}{2}+\frac{q(n)}{2^{n/2}}
                \end{aligned}
            \end{equation}
            考虑以下两种情况:
            \begin{enumerate}
                \item[*] $r$ 曾被使用过,此时 $\AdvA$ 可以确定被加密的消息,但由于 $\AdvA$ 被限制在多项式时间,这种情况的发生次数存在上限 $\frac{q(n)}{2^{n/2}}$。
                \item[*] $r$ 未被使用过,此时 $\AdvA$ 完全无法确定被加密的消息。
            \end{enumerate}
            由此可知:
            \begin{equation}
                \begin{aligned}
                   \Pr[\algo{PrivK}^{cpa}_{\AdvA,\widetilde{\Pi}}(n)=1]=&\Pr[\algo{PrivK}^{cpa}_{\AdvA,\widetilde{\Pi}}(n)=1\land{repeat}]+\Pr[\algo{PrivK}^{cpa}_{\AdvA,\widetilde{\Pi}}(n)=1\land\overline{repeat}]\\
                   &\leq\Pr[repeat]+\Pr[\algo{PrivK}^{cpa}_{\AdvA,\widetilde{\Pi}}(n)=1|\overline{repeat}]\\
                   &\leq\frac{1}{2}+\frac{q(n)}{2^{n/2}}
                \end{aligned}
            \end{equation}
            由式 [\ref{equal-1}] 和 [\ref{equal-2}] 可知:
            \begin{equation}
                \begin{aligned}
                   \Pr[\algo{PrivK}^{cpa}_{\AdvA,\Pi}(n)=1]\leq\frac{1}{2}+\frac{q(n)}{2^{n/2}}+negl(n)
                \end{aligned}
            \end{equation}
            又因为 $\frac{q(n)}{2^{n/2}}$ 同样可忽略,有:
            \begin{equation}
                \begin{aligned}
                   \Pr[\algo{PrivK}^{cpa}_{\AdvA,\Pi}(n)=1]\leq\frac{1}{2}+negl(n)
                \end{aligned}
            \end{equation}
            即方案 $\Pi=(Gen,Enc,Dec)$ 是 CPA 安全的。
        \end{solution}

    \question(3.20题) Let $F$ be a length preserving pseudorandom function and $G$ be a pseudorandom generator with expansion factor $\ell(n)=n+1$. For each of the following encryption schemes, state whether the scheme has indistinguishable encryptions in the presence of an eavesdropper and whether it is CPA-secure.(In each case, the shared key is a uniform $k\in\{0,1\}^n$.) Explain your answer.

        \begin{enumerate}
            \item To encrypt $m\in\{0,1\}^{n+1}$, choose uniform $r\in\{0,1\}^n$ and output the ciphertext $\langle{r,G(r)\oplus{m}}\rangle$.
            \item To encrypt $m\in\{0,1\}^n$, output the ciphertext $m\oplus{F_k}(0^n)$.
            \item To encrypt $m\in\{0,1\}^{2n}$, parse $m$ as $\scriptstyle{m_1}\parallel{m_2}$ with $|m_1|=|m_2|$, then choose uniform $r\in\{0,1\}^n$ and send $\langle{r,m_1\oplus{F_k(r)},m_2\oplus{F_k(r+1)}}\rangle$.
        \end{enumerate}

        \begin{solution}
            \begin{enumerate}
                \item 这个方案与其说是加密不如说是编码,由于并没有用到 $k$,攻击者观察到 $\langle{r,c}\rangle$ 后,只要计算 $G(r)\oplus{c}$,即可直接获得明文。因而这个方案不满足 EAV 安全,进而也不满足 CPA 安全。
                \item 这个方案是 EAV 安全的,只要将 $F_k(0^n)$ 视为伪随机生成器 $G(k)$(假设 $G(k)$ 不是伪随机生成器,可利用对 $G$ 的高效识别器构造对 $F$ 的高效识别器),此时这一方案即为书中的构造 3.17,是 EAV 安全的;另一方面,由于这个方案是确定性的,因而不是 CPA 安全的。
                \item 这个方案是 CPA 安全的,下面进行证明。在此方案 $\Pi=(Gen,Enc,Dec)$ 的基础上构造方案 $\widetilde{\Pi}=(\widetilde{Gen},\widetilde{Enc},\widetilde{Dec})$,$\widetilde{\Pi}$ 除将 $\Pi$ 中使用的伪随机函数 $F$ 替换为真随机函数 $f$ 外与 $\Pi$ 保持一致。\\
                先证明以下等式成立:
                \begin{equation}
                    \label{equal-3}
                    \begin{aligned}
                        |\Pr[\algo{PrivK}^{cpa}_{\AdvA,\Pi}(n)=1]-\Pr[\algo{PrivK}^{cpa}_{\AdvA,\widetilde{\Pi}}(n)=1]|\leq{negl}(n)
                    \end{aligned}
                \end{equation}
                假设上式不成立即存在攻击者 $\AdvA$ 可以高效区分 $\Pi$ 和 $\widetilde{\Pi}$,可按照如下方式构造可高效区分 $F$ 与 $f$ 的区分器 $D$:
                \begin{enumerate}
                    \item 启动 $\AdvA(1^{2n})$,当 $\AdvA$ 向黑箱发出一次查询时,$D$ 随机选取 $r\in\set{0,1}^n$,并将 $m$ 分为等长的 $m_1$ 和 $m_2$ 两部分,分别查询 $F_k(r)$ 和 $F_k(r+1)$ 并计算 $\langle{r,m_1\oplus{F_k(r)},m_2\oplus{F_k(r+1)}}\rangle$ 返回给 $\AdvA$。
                    \item 当 $\AdvA$ 给出消息 $m_0$ 和 $m_1$ 时,随机选择 $b\in\set{0,1}$,$D$ 随机选取 $r\in\set{0,1}^n$,并将 $m_b$ 分为等长的 $m_1$ 和 $m_2$ 两部分,分别查询 $F_k(r)$ 和 $F_k(r+1)$ 并计算 $\langle{r,m_1\oplus{F_k(r)},m_2\oplus{F_k(r+1)}}\rangle$ 返回给 $\AdvA$。
                    \item 按第 1 步持续回答 $\AdvA$ 向黑箱发出查询,直到 $\AdvA$ 输出 $b'$,当 $b'=b$ 时 $D$ 输出 $1$,否则 $D$ 输出 $0$。
                \end{enumerate}
                注意到以下两式:
                \begin{equation}
                    \begin{aligned}
                       \Pr[\algo{PrivK}^{cpa}_{\AdvA,\Pi}(2n)=1]=Pr[D^{F_k(\cdot)}(1^n)=1]
                    \end{aligned}
                \end{equation}
                \begin{equation}
                    \begin{aligned}
                       \Pr[\algo{PrivK}^{cpa}_{\AdvA,\widetilde{\Pi}}(2n)=1]=Pr[D^{f(\cdot)}(1^n)=1]
                    \end{aligned}
                \end{equation}
                由于 $F$ 为伪随机函数,假设不成立,即不存在攻击者 $\AdvA$ 可以高效区分 $\Pi$ 和 $\widetilde{\Pi}$。设 $\AdvA$ 查询黑箱的次数存在多项式上限 $q(\cdot)$,再证明以下等式成立:
                \begin{equation}
                    \label{equal-4}
                    \begin{aligned}
                       \Pr[\algo{PrivK}^{cpa}_{\AdvA,\widetilde{\Pi}}(n)=1]\leq\frac{1}{2}+\frac{q(n)}{2^{n-1}}
                    \end{aligned}
                \end{equation}
                考虑以下两种情况:
                \begin{enumerate}
                    \item[*] $r$ 或 $r-1$ 曾被使用过,此时 $\AdvA$ 可以确定被加密的消息(每一次查询黑盒后,$\AdvA$ 可以确定 $F_k(r)$ 和 $F_k(r + 1)$),但由于 $\AdvA$ 被限制在多项式时间,这种情况的发生次数存在上限 $\frac{2q(n)}{2^n}=\frac{q(n)}{2^{n-1}}$。
                    \item[*] $r$ 和 $r-1$ 未被使用过,此时 $\AdvA$ 完全无法确定被加密的消息。
                \end{enumerate}
                由此可知:
                \begin{equation}
                    \begin{aligned}
                       \Pr[\algo{PrivK}^{cpa}_{\AdvA,\widetilde{\Pi}}(n)=1]=&\Pr[\algo{PrivK}^{cpa}_{\AdvA,\widetilde{\Pi}}(n)=1\land{repeat}]+\Pr[\algo{PrivK}^{cpa}_{\AdvA,\widetilde{\Pi}}(n)=1\land\overline{repeat}]\\
                       &\leq\Pr[repeat]+\Pr[\algo{PrivK}^{cpa}_{\AdvA,\widetilde{\Pi}}(n)=1|\overline{repeat}]\\
                       &\leq\frac{1}{2}+\frac{q(n)}{2^{n-1}}
                    \end{aligned}
                \end{equation}
                由式 [\ref{equal-3}] 和 [\ref{equal-4}] 可知:
                \begin{equation}
                    \begin{aligned}
                       \Pr[\algo{PrivK}^{cpa}_{\AdvA,\Pi}(n)=1]\leq\frac{1}{2}+\frac{q(n)}{2^{n-1}}+negl(n)
                    \end{aligned}
                \end{equation}
                又因为 $\frac{q(n)}{2^{n-1}}$ 同样可忽略,有:
                \begin{equation}
                    \begin{aligned}
                       \Pr[\algo{PrivK}^{cpa}_{\AdvA,\Pi}(n)=1]\leq\frac{1}{2}+negl(n)
                    \end{aligned}
                \end{equation}
                即方案 $\Pi=(Gen,Enc,Dec)$ 是 CPA 安全的,进而也是 EAV 安全的。
            \end{enumerate}
        \end{solution}

    \question (3.25题) Let $F$ be a pseudorandom permutation. Consider the mode of operation in which a uniform value $IV\in\{0,1\}^n$ is chosen, and the $i$th ciphertext block $c_i$ is computed as $c_i:=F_k(IV+i+m_i)$, where addition is modulo $2^n$. Show that this scheme is not EAV-secure.

        \begin{solution}
            \newline
            攻击者可取 $m_0=0^{2n}$,$m_1=0^{n-1}\parallel1\parallel0^n$,注意到以下性质:
            \begin{enumerate}
                \item[*] $m_0$ 将被拆分成 $0^n$ 和 $0^n$ 两部分进行加密,密文分别为 $F_k(IV+i+0^n)$ 和 $F_k(IV+i+1+0^n)$,由于 $F_k$ 是伪随机函数,两部分密文相等的概率可忽略(否则可由此构造对 $F$ 的高效识别器)。
                \item[*] $m_1$ 将被拆分成 $0^{n-1}\parallel1$ 和 $0^n$ 两部分进行加密,密文分别为 $F_k(IV+i+0^{n-1}\parallel1)$ 和 $F_k(IV+i+1+0^n)$,注意到$F_k(IV+i+0^{n-1}\parallel1)=F_k(IV+i+0^n+1)$,两部分密文一定相等。
            \end{enumerate}
            因此,攻击者可根据两部分密文是否相等(相等时输出 $1$)区分被加密的消息是 $m_0$ 还是 $m_1$,上述方案不是 EAV 安全的。
        \end{solution}

    \question ([选做]3.28题) For any function $g:\{0,1\}^n\leftarrow\{0,1\}^n$, define $g^\text{\$}(\cdot)$ to be a $probabilistic$ oracle that, on input $1^n$, chooses uniform $r\in\{0,1\}^n$ and returns $\left\langle{r,g(r)}\right\rangle$. A keyed function $F$ is a $weak\ pseudorandom\ function$ if for all PPT algorithms $D$, there exists a negligible function $\textsf{negl}$ such that:

        \begin{center}
            $\left|\operatorname{Pr}\left[D^{F_k^\$(\cdot)}\left(1^n\right)=1\right]-\operatorname{Pr}\left[D^{f^\$(\cdot)}\left(1^n\right)=1\right]\right|\leq\operatorname{negl}(n)$
        \end{center}

        where $k\in\{0,1\}^n$ and $f\in\textsf{Func}_n$ are chosen uniformly.

        \begin{enumerate}
            \item Prove that $F$ is pseudorandom then it is weakly pseudorandom.
            \item Let $F'$ be a pseudorandom function, and define:
                \begin{center}
                    $F_k(x)\stackrel{\text{def}}{=}\left\{\begin{array}{cl}F_k^{\prime}(x)&\text{if}\ x\ \text{is even}\\F_k^{\prime}(x+1)&\text{if}\ x\ \text{is odd}\end{array}\right.$
                \end{center}
                Prove that $F$ is weakly pseudorandom, but $not$ pseudorandom.
            \item Is CTR-mode encryption using a weak pseudorandom function necessarily CPA-secure? Prove your answer.
            \item Prove that Construction 3.28 is CPA-secure if $F$ is a weak pseudorandom function.
        \end{enumerate}

        \begin{solution}
            \begin{enumerate}
                \item 假设 $F$ 不是弱伪随机函数(存在对 $F$ 的高效识别器 $D'$ 可区分 $F$ 是否为弱伪随机函数),下面说明 $F$ 也不是伪随机函数,可按如下方式构造对 $F$ 的高效识别器 $D$ 以区分 $F$ 是否为伪随机函数:
                    \begin{enumerate}
                        \item[*] 启动对 $F'$ 识别器 $D'$。
                        \item[*] 对于 $D'$ 的每一次询问,$D$ 随机选择 $r\in\{0,1\}^n$ ,并查询 $F_k(r)$,将 $\left\langle{r,F_k(r)}\right\rangle$ 返回给 $D'$。
                        \item[*] 将 $D'$ 的输出作为 $D$ 的输出。
                    \end{enumerate}
                    由于 $F$ 是伪随机函数,假设不成立,$F$ 也是弱伪随机函数。
                \item $F'$ 不是伪随机函数,考虑 $x_1=0^{n-1}\parallel1$ 和 $x_2=0^{n-2}\parallel10$,有 $F_k(x_1)=F_k'(0^{n-2}\parallel10)$ 和 $F_k(x_2)=F_k'(0^{n-2}\parallel10)$;注意到对于真随机的函数 $f$,$f(x_1)$ 和 $f(x_2)$ 相同的概率可忽略,$D$ 可根据函数在 $x_1$ 处的输出和在 $x_2$ 处的输出是否相同(相同时输出 $1$)区分 $F_k$ 和 $f$。\\
                另一方面,$F'$ 是弱伪随机函数,这是因为当且仅当选取了奇数 $i$ 和偶数 $i+1$ 时,才可以区分 $F_k$ 和 $f$,而在 $r$ 为黑箱随机选取而不是由 $D$ 指定的情况下,恰好选取了奇数 $i$ 和偶数 $i+1$ 的概率不超过 $\frac{n}{2^n}$,可忽略,进而区分 $F_k$ 和 $f$ 概率同样可忽略。
                \item 在 CTR 模式下使用弱伪随机函数不满足 CPA 安全。考虑第 2 问中使用的弱伪随机函数 $F_k$,注意到对于奇数 $i$ 和偶数 $i+1$,有 $F_k(i)=F_k(i+1)$,可知第 $i$ 个密文块和第 $i+1$ 个密文块异或的结果等于第 $i$ 个明文块和第 $i+1$ 个明文块异或的结果,攻击者可采取如下策略:
                \begin{enumerate}
                    \item[*] 选择 $m_0=0^{3n}$,若 $b=0$,$m_0$ 将被拆分成 $0^n, 0^n, 0^n$ 三部分进行加密,且第二部分与第三部分的明文异或值、第二部分与第三部分的密文异或值均为 $0^n$。
                    \item[*] 选择 $m_1=0^{2n}\parallel1^n$,若 $b=1$,$m_1$ 将被拆分成 $0^n, 0^n, 1^n$ 三部分进行加密,且第二部分与第三部分的明文异或值、第二部分与第三部分的密文异或值均为 $1^n$。
                \end{enumerate}
                攻击者只要观察第二部分与第三部分的密文异或值(若为 $0^n$ 则输出 $0$),即可区分被加密的消息是 $m_0$ 还是 $m_1$,上述方案不是 EAV 安全的,进而也不是 CPA 安全的。
                \item 在当前方案 $\Pi=(Gen,Enc,Dec)$ 的基础上构造方案 $\widetilde{\Pi}=(\widetilde{Gen},\widetilde{Enc},\widetilde{Dec})$,$\widetilde{\Pi}$ 除将 $\Pi$ 中使用的弱伪随机函数 $F$ 替换为真随机函数 $f$ 外与 $\Pi$ 保持一致。\\
                先证明以下等式成立:
                \begin{equation}
                    \label{equal-5}
                    \begin{aligned}
                        |\Pr[\algo{PrivK}^{cpa}_{\AdvA,\Pi}(n)=1]-\Pr[\algo{PrivK}^{cpa}_{\AdvA,\widetilde{\Pi}}(n)=1]|\leq{negl}(n)
                    \end{aligned}
                \end{equation}
                假设上式不成立即存在攻击者 $\AdvA$ 可以高效区分 $\Pi$ 和 $\widetilde{\Pi}$,可按照如下方式构造可高效区分 $F$ 与 $f$ 的区分器 $D$:
                \begin{enumerate}
                    \item 启动 $\AdvA(1^n)$,当 $\AdvA$ 向黑箱发出一次查询时,$D$ 进行一次查询得到 $\left\langle{r,F_k(r)}\right\rangle$,并将 $\left\langle{r,F_k(r)\oplus{m}}\right\rangle$ 返回给 $\AdvA$。
                    \item 当 $\AdvA$ 给出消息 $m_0$ 和 $m_1$ 时,随机选择 $b\in\set{0,1}$,$D$ 进行一次查询得到 $\left\langle{r,F_k(r)}\right\rangle$,并将 $\left\langle{r,F_k(r)\oplus{m_b}}\right\rangle$ 返回给 $\AdvA$。
                    \item 按第 1 步持续回答 $\AdvA$ 向黑箱发出查询,直到 $\AdvA$ 输出 $b'$,当 $b'=b$ 时 $D$ 输出 $1$,否则 $D$ 输出 $0$。
                \end{enumerate}
                注意到以下两式:
                \begin{equation}
                    \begin{aligned}
                       \Pr[\algo{PrivK}^{cpa}_{\AdvA,\Pi}(n)=1]=Pr[D^{F_k^{\$}(\cdot)}(1^n)=1]
                    \end{aligned}
                \end{equation}
                \begin{equation}
                    \begin{aligned}
                       \Pr[\algo{PrivK}^{cpa}_{\AdvA,\widetilde{\Pi}}(n)=1]=Pr[D^{f^{\$}(\cdot)}(1^n)=1]
                    \end{aligned}
                \end{equation}
                由于 $F$ 为弱伪随机函数,假设不成立,即不存在攻击者 $\AdvA$ 可以高效区分 $\Pi$ 和 $\widetilde{\Pi}$。设 $\AdvA$ 查询黑箱的次数存在多项式上限 $q(\cdot)$,再证明以下等式成立:
                \begin{equation}
                    \label{equal-6}
                    \begin{aligned}
                       \Pr[\algo{PrivK}^{cpa}_{\AdvA,\widetilde{\Pi}}(n)=1]\leq\frac{1}{2}+\frac{q(n)}{2^n}
                    \end{aligned}
                \end{equation}
                考虑以下两种情况:
                \begin{enumerate}
                    \item[*] $r$ 曾被使用过,此时 $\AdvA$ 可以确定被加密的消息,但由于 $\AdvA$ 被限制在多项式时间,这种情况的发生次数存在上限 $\frac{q(n)}{2^n}$。
                    \item[*] $r$ 未被使用过,此时 $\AdvA$ 完全无法确定被加密的消息。
                \end{enumerate}
                由此可知:
                \begin{equation}
                    \begin{aligned}
                       \Pr[\algo{PrivK}^{cpa}_{\AdvA,\widetilde{\Pi}}(n)=1]=&\Pr[\algo{PrivK}^{cpa}_{\AdvA,\widetilde{\Pi}}(n)=1\land{repeat}]+\Pr[\algo{PrivK}^{cpa}_{\AdvA,\widetilde{\Pi}}(n)=1\land\overline{repeat}]\\
                       &\leq\Pr[repeat]+\Pr[\algo{PrivK}^{cpa}_{\AdvA,\widetilde{\Pi}}(n)=1|\overline{repeat}]\\
                       &\leq\frac{1}{2}+\frac{q(n)}{2^n}
                    \end{aligned}
                \end{equation}
                由式 [\ref{equal-5}] 和 [\ref{equal-6}] 可知:
                \begin{equation}
                    \begin{aligned}
                       \Pr[\algo{PrivK}^{cpa}_{\AdvA,\Pi}(n)=1]\leq\frac{1}{2}+\frac{q(n)}{2^n}+negl(n)
                    \end{aligned}
                \end{equation}
                又因为 $\frac{q(n)}{2^n}$ 同样可忽略,有:
                \begin{equation}
                    \begin{aligned}
                       \Pr[\algo{PrivK}^{cpa}_{\AdvA,\Pi}(n)=1]\leq\frac{1}{2}+negl(n)
                    \end{aligned}
                \end{equation}
                即方案 $\Pi=(Gen,Enc,Dec)$ 是 CPA 安全的。
            \end{enumerate}
        \end{solution}

    \question (3.29题) What is the effect of a single bit flip in the ciphertext when using the CBC, OFB, and CTR modes of operation?

        \begin{solution}
            \begin{enumerate}
                \item[*] \textbf{CBC 模式}:注意到 $m_i=F_k^{-1}(c_i)\oplus{c_{i-1}}$,若 $c_i$ 中的某一位出现错误,其对应的明文块 $m_i$ 将完全错误,且接下来的密文块解密后的明文块 $m_{i+1}$ 的一位(对应 $c_i$ 中出错的位)会发生错误。
                \item[*] \textbf{OFB 模式}:注意到用于与密文依次异或的流仅取决于 $c_0$($IV$),若 $c_0$ 中的某一位出现错误,全部密文解密后都将完全错误;若 $c_i$($i>0$)中的某一位出现错误,仅 $c_i$ 对应的明文块 $m_i$ 的对应位会发生错误。
                \item[*] \textbf{CTR 模式}:注意到用于与密文依次异或的流仅取决于 $c_0$($IV$),若 $c_0$ 中的某一位出现错误,全部密文解密后都将完全错误;若 $c_i$($i>0$)中的某一位出现错误,仅 $c_i$ 对应的明文块 $m_i$ 的对应位会发生错误。
            \end{enumerate}
        \end{solution}

    \question (3.30题) What is the effect of a dropped ciphertext block (e.g., if the transmitted ciphertext $c_1,c_2,c_3\cdots$ is received as $c_1,c_3\cdots$) when using the CBC, OFB, and CTR modes of operation?

        \begin{solution}
            \begin{enumerate}
                \item[*] \textbf{CBC 模式}:注意到 $m_i=F_k^{-1}(c_i)\oplus{c_{i-1}}$,所有位于丢失的密文块之前的密文块将正常解密;位于丢失的密文块之后的第一个密文块将错误解密;位于丢失的密文块之后的第二个及之后的密文块将正常解密。
                \item[*] \textbf{OFB 模式}:注意到用于与密文依次异或的流仅取决于 $c_0$($IV$),所有位于丢失的密文块之前的密文块将正常解密;所有位于丢失的密文块之后的密文块将错误解密。
                \item[*] \textbf{CTR 模式}:注意到用于与密文依次异或的流仅取决于 $c_0$($IV$),所有位于丢失的密文块之前的密文块将正常解密;所有位于丢失的密文块之后的密文块将错误解密。
            \end{enumerate}
        \end{solution}

\end{questions}

\end{document}

%%% Local Variables:
%%% mode: latex
%%% TeX-master: t
%%% End:
