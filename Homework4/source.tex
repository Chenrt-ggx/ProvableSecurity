%% HOW TO USE THIS TEMPLATE:
%%
%% Ensure that you replace "YOUR NAME HERE" with your own name, in the
%% \studentname command below.  Also ensure that the "answers" option
%% appears within the square brackets of the \documentclass command,
%% otherwise latex will suppress your solutions when compiling.
%%
%% Type your solution to each problem part within
%% the \begin{solution} \end{solution} environment immediately
%% following it.  Use any of the macros or notation from the
%% header.tex that you need, or use your own (but try to stay
%% consistent with the notation used in the problem).
%%
%% If you have problems compiling this file, you may lack the
%% header.tex file (available on the course web page), or your system
%% may lack some LaTeX packages.  The "exam" package (required) is
%% available at:
%%
%% http://mirror.ctan.org/macros/latex/contrib/exam/exam.cls
%%
%% Other packages can be found at ctan.org, or you may just comment
%% them out (only the exam and ams* packages are absolutely required).

% the "answers" option causes the solutions to be printed
% \documentclass[11pt,addpoints]{exam}

%请使用xelatex编译
\documentclass[11pt,addpoints,answers]{exam}

% required macros -- get header.tex file from course web page
\input{header}

% VARIABLES

\newcommand{\hwnum}{4}
\newcommand{\studentname}{李田所-114514}

% END OF SUPPLIED VARIABLES

\hwheader                       % execute homework commands

\begin{document}

\pagestyle{head}                % put header on every page

% QUESTIONS START HERE.  PROVIDE SOLUTIONS WITHIN THE "solution"
% ENVIRONMENTS FOLLOWING EACH QUESTION.

\begin{questions}
    \question (4.5题) Consider the following MAC for messages of length $\ell(n)=2n-2$ using a pseudorandom function $F$: On input a message $m_0\parallel{m_1}$ (with $|m_0|=|m_1|=n-1$) and key $k\in\set{0,1}^n$, algorithm $\algo{Mac}$ outputs $t=F_k(0\parallel{m_0})\parallel{F_k}(1\parallel{m_1})$. Algorithm $\algo{Vrfy}$ is defined in the natural way. Is this MAC secure? Prove your answer.

        \begin{solution}
            %在此写入答案
        \end{solution}

    \question (4.6题) Let $F$ be a pseudorandom function. Show that each of the following MACs is insecure, even if used to authenticate fixed-length messages. (In each case $\algo{Gen}$ outputs a uniform $k\in\set{0,1}^n$; we let $\langle{i}\rangle$ denote an $n/2$-bit encoding of the integer $i$.)

        \begin{enumerate}
            \item[(a)] To authenticate a message $m=m_1,...,m_\ell$, where $m_i\in\set{0,1}^n$, compute $t:=F_k(m_1)\oplus{...}\oplus{F_k}(m_\ell)$.
            \item[(b)] To authenticate a message $m=m_1,...,m_\ell$, where $m_i\in\set{0,1}^{n/2}$, compute $t:=F_k(\langle{1}\rangle\parallel{m_1})\oplus{...}\oplus{F_k}(\langle\ell\rangle\parallel{m_\ell})$.
            \item[(c)] To authenticate a message $m=m_1,...,m_\ell$, where $m_i\in\set{0,1}^{n/2}$, choose uniform $r\in\set{0,1}^n$, compute:
                \begin{center}
                    $t:=F_k(r)\oplus{F_k}(\langle{1}\rangle\parallel{m_1})\oplus{...}\oplus{F_k}(\langle\ell\rangle\parallel{m_\ell})$
                \end{center}
                and let the tag be $\langle{r,t}\rangle$.
        \end{enumerate}

        \begin{solution}
            %在此写入答案
        \end{solution}

    \question (4.7题) Let $F$ be a pseudorandom function. Show that the following MACsfor messages of length $2n$ is insecure: $\algo{Gen}$ outputs a uniform $k\in\set{0,1}^n$. To authenticate a message $m_1\parallel{m_2}$ with $|m_1|=|m_2|=n$, compute the tag $F_k(m_1)\parallel{F_k}(F_k(m_2))$.

        \begin{solution}
            %在此写入答案
        \end{solution}

    \question (4.8题) Given any $deterministic$ MAC$(\algo{Mac},\algo{Vrfy})$, we may view $\algo{Mac}$ as a keyed function. In both Constructions 4.5 and 4.9, $\algo{Mac}$ is a pseudorandom function. Give a construction of a secure, deterministic MAC in which Mac is $not$ a pseudorandom function.

        \begin{solution}
            %在此写入答案
        \end{solution}

    \question (4.13题) We explore what happens when the basic CBC-MAC construction is used with messages of different lengths.

        \begin{enumerate}
            \item[(a)] Say the sender and receiver do not agree on the message length in advance (and so $\algo{Vrfy_k}(m,t)=1$ iff $t\overset{?}{=}\algo{Mac_k}(m)$, regardless of the length of $m$), but the sender is careful to only authenticate messages of length $2n$. Show that an adversary can forge a valid tag on a message of length $4n$.
            \item[(b)] Say the receiver only accepts 3-block messages (so $\algo{Vrfy_k}(m,t)=1$ only if $m$ has length $3n$ and $t\overset{?}{=}\algo{Mac_k}(m)$), but the sender authenticates messages of any length a multiple of $n$. Show that an adversary can forge a valid tag on a new message.
        \end{enumerate}

        \begin{solution}
            %在此写入答案
        \end{solution}

    \question (4.14题) Prove that the following modifications of basic CBC-MAC do not yield a secure MAC (even for fixed-length messages):

        \begin{enumerate}
            \item[(a)] Mac outputs all blocks $t_1,\dots,t_\ell$, rather than just $t_\ell$. (Verification only checks whether $t_\ell$ is correct.)
            \item[(b)] A random initial block is used each time a message is authenticated. That is, change Construction 4.9 by choosing uniform $t_0\in\{0,1\}^n$, computing $t_\ell$ as before, and then outputting the tag $\langle{t_0,t_\ell}\rangle$; verification is done in the natural way.
        \end{enumerate}

        \begin{solution}
            %在此写入答案
        \end{solution}

    \question (4.15题)[选做] Show that appending the message length to the end of the message before applying basic CBC-MAC does not result in a secure MAC for arbitrary-length messages.

        \begin{solution}
            %在此写入答案
        \end{solution}

    \question (4.16题)[选做] Define a version of CBC-MAC for messages of length at most $\ell\cdot{2^n}$ as follows: given a message $m$, pad it with $0$s so that it has length exactly $\ell\cdot{2^n}$; apply basic CBC-MAC to the result. Is this secure?

        \begin{solution}
            %在此写入答案
        \end{solution}

    \question (4.18题)[选做] Prove that the encoding for arbitrary-length messages described in Section 4.4.2 is prefix-free.

        \begin{solution}
            %在此写入答案
        \end{solution}

\end{questions}

\end{document}

%%% Local Variables:
%%% mode: latex
%%% TeX-master: t
%%% End:
