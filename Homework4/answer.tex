%% HOW TO USE THIS TEMPLATE:
%%
%% Ensure that you replace "YOUR NAME HERE" with your own name, in the
%% \studentname command below.  Also ensure that the "answers" option
%% appears within the square brackets of the \documentclass command,
%% otherwise latex will suppress your solutions when compiling.
%%
%% Type your solution to each problem part within
%% the \begin{solution} \end{solution} environment immediately
%% following it.  Use any of the macros or notation from the
%% header.tex that you need, or use your own (but try to stay
%% consistent with the notation used in the problem).
%%
%% If you have problems compiling this file, you may lack the
%% header.tex file (available on the course web page), or your system
%% may lack some LaTeX packages.  The "exam" package (required) is
%% available at:
%%
%% http://mirror.ctan.org/macros/latex/contrib/exam/exam.cls
%%
%% Other packages can be found at ctan.org, or you may just comment
%% them out (only the exam and ams* packages are absolutely required).

% the "answers" option causes the solutions to be printed
% \documentclass[11pt,addpoints]{exam}

%请使用xelatex编译
\documentclass[11pt,addpoints,answers]{exam}

% required macros -- get header.tex file from course web page
% !Mode:: "TeX:UTF-8"
% !TEX root=./hw1.tex
\usepackage{xeCJK}    % 写中文要用到
\usepackage{tikz}
\usepackage{amsmath,amsfonts,amssymb,amsthm}
\usepackage{fullpage}
\usepackage{times}
\usepackage{hyperref}
\usepackage{pdfsync}
\usepackage{microtype}
\usepackage{color}
\usepackage{cleveref}
\crefformat{footnote}{#2\footnotemark[#1]#3}
\definecolor{light-gray}{gray}{0.5}

\renewcommand{\tablename}{表}
\renewcommand{\figurename}{图}
\renewcommand{\proof}{证明}

% \theoremstyle{plain}
% \newtheorem{theorem}{定理~}
% \newtheorem{lemma}{引理~}
% \newtheorem{axiom}{公理~}
% \newtheorem{proposition}{命题~}
% \newtheorem{prop}{性质~}
% \newtheorem{corollary}{推论~}
% \newtheorem{conclusion}{结论~}
% \newtheorem{definition}{定义~}
% \newtheorem{conjecture}{猜想~}
% \newtheorem{example}{例~}
% \newtheorem{remark}{注~}
% \newtheorem{algorithm}{算法~}

%%% BLACKBOARD SYMBOLS

% \newcommand{\C}{\ensuremath{\mathbb{C}}}
\newcommand{\D}{\ensuremath{\mathbb{D}}}
\newcommand{\F}{\ensuremath{\mathbb{F}}}
% \newcommand{\G}{\ensuremath{\mathbb{G}}}
\newcommand{\J}{\ensuremath{\mathbb{J}}}
\newcommand{\N}{\ensuremath{\mathbb{N}}}
\newcommand{\Q}{\ensuremath{\mathbb{Q}}}
\newcommand{\R}{\ensuremath{\mathbb{R}}}
\newcommand{\T}{\ensuremath{\mathbb{T}}}
\newcommand{\Z}{\ensuremath{\mathbb{Z}}}
\newcommand{\QR}{\ensuremath{\mathbb{QR}}}

\newcommand{\Zt}{\ensuremath{\Z_t}}
\newcommand{\Zp}{\ensuremath{\Z_p}}
\newcommand{\Zq}{\ensuremath{\Z_q}}
\newcommand{\ZN}{\ensuremath{\Z_N}}
\newcommand{\Zps}{\ensuremath{\Z_p^*}}
\newcommand{\ZNs}{\ensuremath{\Z_N^*}}
\newcommand{\JN}{\ensuremath{\J_N}}
\newcommand{\QRN}{\ensuremath{\QR_{N}}}
\newcommand{\QRp}{\ensuremath{\QR_{p}}}

%%% THEOREM COMMANDS

% following are "theorem" style
\theoremstyle{plain}
\newtheorem{theorem}{定理}[section]
\newtheorem{lemma}[theorem]{引理}
\newtheorem{corollary}[theorem]{推论}
\newtheorem{proposition}[theorem]{命题}
\newtheorem{claim}[theorem]{声明}
\newtheorem{fact}[theorem]{事实}

% following are def style
\theoremstyle{definition}
\newtheorem{definition}[theorem]{定义}
\newtheorem{conjecture}[theorem]{推测}
\newtheorem{example}[theorem]{例}
\newtheorem{protocol}[theorem]{协议}

% following are remark style
\theoremstyle{remark}
\newtheorem{remark}[theorem]{注}
\newtheorem{note}[theorem]{注}
\newtheorem{exercise}[theorem]{练习}

% equation numbering style
\numberwithin{equation}{section}

%%% GENERAL COMPUTING

\newcommand{\bit}{\ensuremath{\set{0,1}}}
\newcommand{\pmone}{\ensuremath{\set{-1,1}}}

% asymptotics
\DeclareMathOperator{\poly}{poly}
\DeclareMathOperator{\polylog}{polylog}
\DeclareMathOperator{\negl}{negl}
\newcommand{\Otil}{\ensuremath{\tilde{O}}}

% probability/distribution stuff
\DeclareMathOperator*{\E}{E}
\DeclareMathOperator*{\Var}{Var}

% sets in calligraphic type
\newcommand{\calA}{\ensuremath{\mathcal{A}}}
\newcommand{\calD}{\ensuremath{\mathcal{D}}}
\newcommand{\calF}{\ensuremath{\mathcal{F}}}
\newcommand{\calH}{\ensuremath{\mathcal{H}}}
\newcommand{\calK}{\ensuremath{\mathcal{K}}}
\newcommand{\calM}{\ensuremath{\mathcal{M}}}
\newcommand{\calX}{\ensuremath{\mathcal{X}}}
\newcommand{\calY}{\ensuremath{\mathcal{Y}}}

% types of indistinguishability
\newcommand{\compind}{\ensuremath{\stackrel{c}{\approx}}}
\newcommand{\statind}{\ensuremath{\stackrel{s}{\approx}}}
\newcommand{\perfind}{\ensuremath{\equiv}}

% font for general-purpose algorithms
\newcommand{\algo}[1]{\ensuremath{\mathsf{#1}}}
% font for general-purpose computational problems
\newcommand{\problem}[1]{\ensuremath{\mathsf{#1}}}
% font for complexity classes
\newcommand{\class}[1]{\ensuremath{\mathsf{#1}}}

% complexity classes and languages
\renewcommand{\P}{\class{P}}
\newcommand{\BPP}{\class{BPP}}
\newcommand{\NP}{\class{NP}}
\newcommand{\coNP}{\class{coNP}}
\newcommand{\AM}{\class{AM}}
\newcommand{\coAM}{\class{coAM}}
\newcommand{\IP}{\class{IP}}

%%% "LEFT-RIGHT" PAIRS OF SYMBOLS

% inner product
\newcommand{\inner}[1]{\langle{#1}\rangle}
\newcommand{\innerfit}[1]{\left\langle{#1}\right\rangle}
% absolute value
\newcommand{\abs}[1]{\lvert{#1}\rvert}
\newcommand{\absfit}[1]{\left\lvert{#1}\right\rvert}
% a set
\newcommand{\set}[1]{\{{#1}\}}
\newcommand{\setfit}[1]{\left\{{#1}\right\}}
% parens
\newcommand{\parens}[1]{({#1})}
\newcommand{\parensfit}[1]{\left({#1}\right)}
% tuple = alias for parens
\newcommand{\tuple}[1]{\parens{#1}}
\newcommand{\tuplefit}[1]{\parensfit{#1}}
% square brackets
\newcommand{\bracks}[1]{[{#1}]}
\newcommand{\bracksfit}[1]{\left[{#1}\right]}
% rounding off
\newcommand{\round}[1]{\lfloor{#1}\rceil}
% floor function
\newcommand{\floor}[1]{\lfloor{#1}\rfloor}
% ceiling function
\newcommand{\ceil}[1]{\lceil{#1}\rceil}
% length of a string
\newcommand{\len}[1]{\lvert{#1}\rvert}
\newcommand{\lenfit}[1]{\left\lvert{#1}\right\rvert}
% length of some vector, element
\newcommand{\length}[1]{\lVert{#1}\rVert}
\newcommand{\lengthfit}[1]{\left\lVert{#1}\right\rVert}

%%% CRYPTO-RELATED NOTATION

% KEYS AND RELATED

\newcommand{\key}[1]{\ensuremath{#1}}

\newcommand{\pk}{\key{pk}}
\newcommand{\vk}{\key{vk}}
\newcommand{\sk}{\key{sk}}
\newcommand{\mpk}{\key{mpk}}
\newcommand{\msk}{\key{msk}}
\newcommand{\fk}{\key{fk}}
\newcommand{\id}{id}
\newcommand{\keyspace}{\ensuremath{\mathcal{K}}}
\newcommand{\msgspace}{\ensuremath{\mathcal{M}}}
\newcommand{\ctspace}{\ensuremath{\mathcal{C}}}
\newcommand{\tagspace}{\ensuremath{\mathcal{T}}}
\newcommand{\idspace}{\ensuremath{\mathcal{ID}}}
\newcommand{\concat}{\ensuremath{\|}}

% GAMES

% advantage
\newcommand{\advan}{\ensuremath{\mathbf{Adv}}}

% different attack models
\newcommand{\attack}[1]{\ensuremath{\text{#1}}}

\newcommand{\atk}{\attack{atk}}        % dummy attack
\newcommand{\indcpa}{\attack{ind-cpa}}
\newcommand{\indcca}{\attack{ind-cca}}
\newcommand{\anocpa}{\attack{ano-cpa}} % anonymous
\newcommand{\anocca}{\attack{ano-cca}}
\newcommand{\euacma}{\attack{eu-acma}} % forgery: adaptive chosen-message
\newcommand{\euscma}{\attack{eu-scma}} % forgery: static chosen-message
\newcommand{\suacma}{\attack{su-acma}} % strongly unforgeable

% ADVERSARIES

\newcommand{\attacker}[1]{\ensuremath{\mathcal{#1}}}

\newcommand{\Adv}{\attacker{A}}
\newcommand{\AdvA}{\attacker{A}}
\newcommand{\AdvB}{\attacker{B}}
\newcommand{\Dist}{\attacker{D}}
\newcommand{\Sim}{\attacker{S}}
\newcommand{\Ora}{\attacker{O}}
\newcommand{\Inv}{\attacker{I}}
\newcommand{\For}{\attacker{F}}

% CRYPTO SCHEMES

\newcommand{\scheme}[1]{\ensuremath{\text{#1}}}

% pseudorandom stuff
\newcommand{\prg}{\algo{PRG}}
\newcommand{\prf}{\algo{PRF}}
\newcommand{\prp}{\algo{PRP}}

% symmetric-key cryptosystem
\newcommand{\skc}{\scheme{SKC}}
\newcommand{\skcgen}{\algo{Gen}}
\newcommand{\skcenc}{\algo{Enc}}
\newcommand{\skcdec}{\algo{Dec}}

% public-key cryptosystem
\newcommand{\pkc}{\scheme{PKC}}
\newcommand{\pkcgen}{\algo{Gen}}
\newcommand{\pkcenc}{\algo{Enc}}
\newcommand{\pkcdec}{\algo{Dec}}

% digital signatures
\newcommand{\sig}{\scheme{SIG}}
\newcommand{\siggen}{\algo{Gen}}
\newcommand{\sigsign}{\algo{Sign}}
\newcommand{\sigver}{\algo{Ver}}

% message authentication code
\newcommand{\mac}{\scheme{MAC}}
\newcommand{\macgen}{\algo{Gen}}
\newcommand{\mactag}{\algo{Tag}}
\newcommand{\macver}{\algo{Ver}}

% key-encapsulation mechanism
\newcommand{\kem}{\scheme{KEM}}
\newcommand{\kemgen}{\algo{Gen}}
\newcommand{\kemenc}{\algo{Encaps}}
\newcommand{\kemdec}{\algo{Decaps}}

% identity-based encryption
\newcommand{\ibe}{\scheme{IBE}}
\newcommand{\ibesetup}{\algo{Setup}}
\newcommand{\ibeext}{\algo{Ext}}
\newcommand{\ibeenc}{\algo{Enc}}
\newcommand{\ibedec}{\algo{Dec}}

% hierarchical IBE (as key encapsulation)
\newcommand{\hibe}{\scheme{HIBE}}
\newcommand{\hibesetup}{\algo{Setup}}
\newcommand{\hibeext}{\algo{Extract}}
\newcommand{\hibeenc}{\algo{Encaps}}
\newcommand{\hibedec}{\algo{Decaps}}

% binary tree encryption (as key encapsulation)
\newcommand{\bte}{\scheme{BTE}}
\newcommand{\btesetup}{\algo{Setup}}
\newcommand{\bteext}{\algo{Extract}}
\newcommand{\bteenc}{\algo{Encaps}}
\newcommand{\btedec}{\algo{Decaps}}

% trapdoor functions
\newcommand{\tdf}{\scheme{TDF}}
\newcommand{\tdfgen}{\algo{Gen}}
\newcommand{\tdfeval}{\algo{Eval}}
\newcommand{\tdfinv}{\algo{Invert}}
\newcommand{\tdfver}{\algo{Ver}}

%%% PROTOCOLS

\newcommand{\out}{\text{out}}
\newcommand{\view}{\text{view}}

%%% COMMANDS FOR LECTURES/HOMEWORKS

\newcommand{\lecheader}{
\chead{
    \large \textbf{Lecture \lecturenum\\\lecturetopic}
}

\lhead{\small
    \textbf{\href{http://bhpan.buaa.edu.cn}{可证明安全技术}\\2022年秋季}
}

\rhead{\small
    \textbf{主讲: 张宗洋
}

\setlength{\headheight}{20pt}
\setlength{\headsep}{16pt}
}}

\newcommand{\hwheader}{
\chead{
    \Large \textbf{作业 \hwnum}
}

\lhead{\small
    \textbf{{可证明安全技术}\\2022年秋季}
}

\rhead{\small
    \textbf{主讲: 张宗洋\\姓名: \studentname}
}

\setlength{\headheight}{20pt}
\setlength{\headsep}{16pt}
\headrule
}

\newcommand{\examheader}{
\chead{
    \Large \textbf{测试 \\ \duedate}
}

\lhead{\small
    \textbf{\href{http://bhpan.buaa.edu.cn}{Introduction to modern cryptography}\\2022年秋季}}
    \rhead{\small \textbf{主讲: 张宗洋\\姓名: \studentname}}
    \setlength{\headheight}{20pt}
    \setlength{\headsep}{16pt}
    \headrule
}

\newcommand{\hint}[1]{\href{http://www.cims.nyu.edu/~regev/cgi-bin/hints/index.html?#1}{{\textcolor{light-gray}{I need a hint! (ID #1)}}}}

\newcommand{\hinttext}[2]{\href{http://www.cims.nyu.edu/~regev/cgi-bin/hints/index.html?#1}{{\textcolor{light-gray}{#2 (ID #1)}}}}

\newcommand{\hintcost}[2]{\href{http://www.cims.nyu.edu/~regev/cgi-bin/hints/index.html?#1}{{\textcolor{light-gray}{I need a hint for #2 points! (ID #1)}}}}

\newcommand{\moreinfo}[1]{\href{http://www.cims.nyu.edu/~regev/cgi-bin/hints/index.html?#1}{{\textcolor{light-gray}{I'm done solving and want to know more! (ID #1)}}}}


% VARIABLES

\newcommand{\hwnum}{4}
\newcommand{\studentname}{李田所-114514}

% END OF SUPPLIED VARIABLES

\hwheader                       % execute homework commands

\begin{document}

\pagestyle{head}                % put header on every page

% QUESTIONS START HERE.  PROVIDE SOLUTIONS WITHIN THE "solution"
% ENVIRONMENTS FOLLOWING EACH QUESTION.

\begin{questions}
    \question (4.5题) Consider the following MAC for messages of length $\ell(n)=2n-2$ using a pseudorandom function $F$: On input a message $m_0\parallel{m_1}$ (with $|m_0|=|m_1|=n-1$) and key $k\in\set{0,1}^n$, algorithm $\algo{Mac}$ outputs $t=F_k(0\parallel{m_0})\parallel{F_k}(1\parallel{m_1})$. Algorithm $\algo{Vrfy}$ is defined in the natural way. Is this MAC secure? Prove your answer.

        \begin{solution}
            \newline
            此方案不是安全的 MAC,攻击者可采用如下策略:
            \begin{enumerate}
                \item[*] 构造 $m_1=m_a\parallel{m_b}$ 和 $m_2=m_c\parallel{m_d}$,并向 Oracle 询问得到 $t_1=t_a\parallel{t_b}$ 和 $t_2=t_c\parallel{t_d}$,其中 $m_a\neq{m_c}$ 且 $m_b\neq{m_d}$。
                \item[*] 输出 $m=m_a\parallel{m_d}$ 和 $t=t_a\parallel{t_d}$(也可以输出 $m=m_b\parallel{m_c}$ 和 $t=t_b\parallel{t_c}$,下以 $m=m_a\parallel{m_d}$ 为例)。
            \end{enumerate}
            注意到 $t=F_k(0\parallel{m_a})\parallel{F_k}(1\parallel{m_d})$,而:
            \begin{enumerate}
                \item[*] $t_a=F_k(0\parallel{m_a})$、$t_b=F_k(1\parallel{m_b})$,
                \item[*] $t_c=F_k(0\parallel{m_c})$、$t_d=F_k(1\parallel{m_d})$。
            \end{enumerate}
            因而,$\algo{Vrfy}_k(m,t)=1$,另一方面,显然 $m_a\parallel{m_d}$ 没有被查询过,即攻击者有不可忽略的概率攻击成功。
        \end{solution}

    \question (4.6题) Let $F$ be a pseudorandom function. Show that each of the following MACs is insecure, even if used to authenticate fixed-length messages. (In each case $\algo{Gen}$ outputs a uniform $k\in\set{0,1}^n$; we let $\langle{i}\rangle$ denote an $n/2$-bit encoding of the integer $i$.)

        \begin{enumerate}
            \item[(a)] To authenticate a message $m=m_1,...,m_\ell$, where $m_i\in\set{0,1}^n$, compute $t:=F_k(m_1)\oplus{...}\oplus{F_k}(m_\ell)$.
            \item[(b)] To authenticate a message $m=m_1,...,m_\ell$, where $m_i\in\set{0,1}^{n/2}$, compute $t:=F_k(\langle{1}\rangle\parallel{m_1})\oplus{...}\oplus{F_k}(\langle\ell\rangle\parallel{m_\ell})$.
            \item[(c)] To authenticate a message $m=m_1,...,m_\ell$, where $m_i\in\set{0,1}^{n/2}$, choose uniform $r\in\set{0,1}^n$, compute:
                \begin{center}
                    $t:=F_k(r)\oplus{F_k}(\langle{1}\rangle\parallel{m_1})\oplus{...}\oplus{F_k}(\langle\ell\rangle\parallel{m_\ell})$
                \end{center}
                and let the tag be $\langle{r,t}\rangle$.
        \end{enumerate}

        \begin{solution}
            \begin{enumerate}
                \item[(a)] 攻击者可采用如下策略:选择不相同的 $m_1$ 和 $m_2$,并向 Oracle 询问 $m'=m_1\parallel{m_2}$ 得到 $t'=F_k(m_1)\oplus{F_k}(m_2)$,输出 $m=m_2\parallel{m_1}$ 和 $t=t'$。\\
                注意到 $F_k(m_1)\oplus{F_k}(m_2)=F_k(m_2)\oplus{F_k}(m_1)$,有 $\algo{Vrfy}_k(m,t)=1$,另一方面,显然 $m_2\parallel{m_1}$ 没有被查询过,即攻击者有不可忽略的概率攻击成功。
                \item[(b)] 攻击者可采用如下策略:
                    \begin{enumerate}
                        \item[*] 构造互不相同的 $m_a, m_b, m_c, m_d$,向 Oracle 询问 $m_1=m_a\parallel{m_b}$ 得到 $t_1=t_a\parallel{t_b}$、询问 $m_2=m_a\parallel{m_c}$ 得到 $t_2=t_a\parallel{t_c}$、询问 $m_3=m_b\parallel{m_c}$ 得到 $t_3=t_b\parallel{t_c}$。
                        \item[*] 输出 $m=m_b\parallel{m_d}$ 和 $t=t_1\oplus{t_2}\oplus{t_3}$。
                    \end{enumerate}
                    注意到 $t=F_k(\langle{1}\rangle\parallel{m_b})\oplus{F_k(\langle{2}\rangle\parallel{m_d})}$,而:
                    \begin{enumerate}
                        \item[*] $t_1=F_k(\langle{1}\rangle\parallel{m_a})\oplus{F_k(\langle{2}\rangle\parallel{m_b})}$,
                        \item[*] $t_2=F_k(\langle{1}\rangle\parallel{m_a})\oplus{F_k(\langle{2}\rangle\parallel{m_c})}$,
                        \item[*] $t_3=F_k(\langle{1}\rangle\parallel{m_b})\oplus{F_k(\langle{2}\rangle\parallel{m_c})}$。
                    \end{enumerate}
                    因而有 $t_1\oplus{t_2}\oplus{t_3}=F_k(\langle{1}\rangle\parallel{m_b})\oplus{F_k(\langle{2}\rangle\parallel{m_d})}$,故 $\algo{Vrfy}_k(m,t)=1$,另一方面,显然 $m_b\parallel{m_d}$ 没有被查询过,即攻击者有不可忽略的概率攻击成功。
                \item[(c)] 攻击者可采用如下策略:任意选择 $m'$,直接输出 $m=m'$ 和 $t=\langle\langle{1}\rangle\parallel{m'},0^n\rangle$。\\
                注意到 $t=F_k(r)\oplus{F_k}(\langle{1}\rangle\parallel{m'})=F_k(\langle{1}\rangle\parallel{m'})\oplus{F_k}(\langle{1}\rangle\parallel{m'})=0^n$,有 $\algo{Vrfy}_k(m,t)=1$,另一方面,显然 $m'$ 没有被查询过,即攻击者有不可忽略的概率攻击成功。
            \end{enumerate}
        \end{solution}

    \question (4.7题) Let $F$ be a pseudorandom function. Show that the following MACsfor messages of length $2n$ is insecure: $\algo{Gen}$ outputs a uniform $k\in\set{0,1}^n$. To authenticate a message $m_1\parallel{m_2}$ with $|m_1|=|m_2|=n$, compute the tag $F_k(m_1)\parallel{F_k}(F_k(m_2))$.

        \begin{solution}
            \newline
            攻击者可采用如下策略:
            \begin{enumerate}
                \item 构造 $m_1=m_a\parallel{m_b}$ 和 $m_2=m_c\parallel{m_d}$,并向 Oracle 询问得到 $t_1=t_a\parallel{t_b}$ 和 $t_2=t_c\parallel{t_d}$,其中 $m_a\neq{m_c}$ 且 $m_b\neq{m_d}$。
                \item 输出 $m=m_a\parallel{m_d}$ 和 $t=t_a\parallel{t_d}$(也可以输出 $m=m_b\parallel{m_c}$ 和 $t=t_b\parallel{t_c}$,下以 $m=m_a\parallel{m_d}$ 为例)。
            \end{enumerate}
            注意到 $t=F_k(m_a)\parallel{F_k}(F_k(m_d))$,而:
            \begin{enumerate}
                \item[*] $t_a=F_k(m_a)$、$t_b=F_k(F_k(m_b))$,
                \item[*] $t_c=F_k(m_c)$、$t_d=F_k(F_k(m_d))$。
            \end{enumerate}
            因而,$\algo{Vrfy}_k(m,t)=1$,另一方面,显然 $m_a\parallel{m_d}$ 没有被查询过,即攻击者有不可忽略的概率攻击成功。
        \end{solution}

    \question (4.8题) Given any $deterministic$ MAC$(\algo{Mac},\algo{Vrfy})$, we may view $\algo{Mac}$ as a keyed function. In both Constructions 4.5 and 4.9, $\algo{Mac}$ is a pseudorandom function. Give a construction of a secure, deterministic MAC in which Mac is $not$ a pseudorandom function.

        \begin{solution}
            \newline
            对于伪随机函数 $F$,构造如下方案:
            \begin{enumerate}
                \item[\algo{Mac}:] 对于输入的 $k\in\set{0,1}^n$ 和消息 $m\in\set{0,1}^n$,输出认证码 $t:=1\parallel{F_k}(m)$。
                \item[\algo{Vrfy}:] 对于输入的 $k\in\set{0,1}^n$ 和消息 $m\in\set{0,1}^n$、认证码 $t\in\set{0,1}^{n+1}$,当且仅当 $t=1\parallel{F_k}(m)$ 时输出 $1$。
            \end{enumerate}
            上述做法确定性和正确性显然,且对于随机选取的 $t$,$t$ 以 $1$ 开始的概率为 $\frac{1}{2}$,而对于按照上述 MAC 方案产生的 $t$,$t$ 以 $1$ 开始的概率为 $1$,因而上述 MAC 方案不是伪随机函数。
        \end{solution}

    \question (4.13题) We explore what happens when the basic CBC-MAC construction is used with messages of different lengths.

        \begin{enumerate}
            \item[(a)] Say the sender and receiver do not agree on the message length in advance (and so $\algo{Vrfy_k}(m,t)=1$ iff $t\overset{?}{=}\algo{Mac_k}(m)$, regardless of the length of $m$), but the sender is careful to only authenticate messages of length $2n$. Show that an adversary can forge a valid tag on a message of length $4n$.
            \item[(b)] Say the receiver only accepts 3-block messages (so $\algo{Vrfy_k}(m,t)=1$ only if $m$ has length $3n$ and $t\overset{?}{=}\algo{Mac_k}(m)$), but the sender authenticates messages of any length a multiple of $n$. Show that an adversary can forge a valid tag on a new message.
        \end{enumerate}

        \begin{solution}
            \begin{enumerate}
                \item[(a)] 攻击者可采用如下策略:
                \begin{enumerate}
                    \item[*] 构造 $m'=m_1\parallel{m_2}$ 并向 Oracle 询问得到 $t'=F_k(F_k(m_1)\oplus{m_2})$。
                    \item[*] 输出 $m=m_1\parallel{m_2}\parallel{t'\oplus{m_1}}\parallel{m_2}$ 和 $t=t'$。
                \end{enumerate}
                注意到对于攻击者构造的 $m$,有:
                \begin{equation}
                    \begin{aligned}
                        t=&F_k(F_k(F_k(F_k(m_1)\oplus{m_2})\oplus{(t'\oplus{m_1})})\oplus{m_2})\\
                        =&F_k(F_k(F_k(F_k(m_1)\oplus{m_2})\oplus{(F_k(F_k(m_1)\oplus{m_2})\oplus{m_1})})\oplus{m_2})\\
                        =&F_k(F_k(m_1)\oplus{m_2})
                    \end{aligned}
                \end{equation}
                因而 $\algo{Vrfy}_k(m,t)=1$,另一方面,显然 $m=m_1\parallel{m_2}\parallel{t\oplus{m_1}}\parallel{m_2}$ 没有被查询过,即攻击者有不可忽略的概率攻击成功。
                \item[(b)] 攻击者可采用如下策略:
                \begin{enumerate}
                    \item[*] 随机选择 $m_1$ 并向 Oracle 询问 $m_1$ 得到 $t_1$。
                    \item[*] 随机选择 $m_2$ 并向 Oracle 询问 $m_2\oplus{t_1}$ 得到 $t_2$。
                    \item[*] 随机选择 $m_3$ 并向 Oracle 询问 $m_3\oplus{t_2}$ 得到 $t_3$。
                    \item[*] 输出 $m=m_1\parallel{m_2}\parallel{m_3}$ 和 $t=t_3$。
                \end{enumerate}
                注意到对于攻击者构造的 $m$,有:
                \begin{equation}
                    \begin{aligned}
                        t=&F_k(F_k(F_k(m_1)\oplus{m_2})\oplus{m_3})\\
                        =&F_k(F_k(t_1\oplus{m_2})\oplus{m_3})\\
                        =&F_k(t_2\oplus{m_3})=t_3
                    \end{aligned}
                \end{equation}
                因而 $\algo{Vrfy}_k(m,t)=1$,另一方面,显然 $m=m_1\parallel{m_2}\parallel{m_3}$ 没有被查询过,即攻击者有不可忽略的概率攻击成功。
            \end{enumerate}
        \end{solution}

    \question (4.14题) Prove that the following modifications of basic CBC-MAC do not yield a secure MAC (even for fixed-length messages):

        \begin{enumerate}
            \item[(a)] Mac outputs all blocks $t_1,\dots,t_\ell$, rather than just $t_\ell$. (Verification only checks whether $t_\ell$ is correct.)
            \item[(b)] A random initial block is used each time a message is authenticated. That is, change Construction 4.9 by choosing uniform $t_0\in\{0,1\}^n$, computing $t_\ell$ as before, and then outputting the tag $\langle{t_0,t_\ell}\rangle$; verification is done in the natural way.
        \end{enumerate}

        \begin{solution}
            \begin{enumerate}
                \item[(a)] 攻击者可采用如下策略:构造 $m'=m_1\parallel{m_2}$ 并向 Oracle 询问得到 $t_1'=F_k(m_1)$ 和 $t_2'=F_k(F_k(m_1)\oplus{m_2})$,输出 $m=m_2\oplus{t_1'}\parallel{m_1}\oplus{t_2'}$ 和 $t_1=t_2'$、$t_2=t_1'$。\\
                注意到对于攻击者构造的 $m$,有:
                \begin{equation}
                    \begin{aligned}
                        t_1=&F_k(m_2\oplus{t_1'})=F_k(m_2\oplus{F_k(m_1)})\\
                        t_2=&F_k(t_2'\oplus{m_1}\oplus{F_k(m_2\oplus{t_1'})})\\
                        =&F_k(F_k(F_k(m_1)\oplus{m_2})\oplus{m_1}\oplus{F_k(m_2\oplus{F_k(m_1)})})=F_k(m_1)
                    \end{aligned}
                \end{equation}
                因而 $\algo{Vrfy}_k(m,t)=1$,另一方面,显然 $m=m_2\oplus{t_1'}\parallel{m_1}\oplus{t_2'}$ 被查询过的概率($m_1=m_2\oplus{t_1'}$ 和 $m_2=m_1\oplus{t_2'}$ )可忽略,即攻击者有不可忽略的概率攻击成功。
                \item[(b)] 攻击者可采用如下策略:随机选择 $m'$ 并向 Oracle 询问得到 $\langle{t_0,t_1}\rangle$,再直接输出 $m=t_0$ 和 $t=\langle{m',t_1}\rangle$。\\
                注意到对于攻击者构造的 $m$,有 $t_1=F_k(m'\oplus{t_0})$,因而 $\algo{Vrfy}_k(m,t)=1$,另一方面,显然 $m=t_0$ 被查询过的概率($m=t_0=m'$,$m$ 和 $t_0$ 为随机选择)可忽略,即攻击者有不可忽略的概率攻击成功。
            \end{enumerate}
        \end{solution}

    \question (4.15题)[选做] Show that appending the message length to the end of the message before applying basic CBC-MAC does not result in a secure MAC for arbitrary-length messages.

        \begin{solution}
            \newline
            攻击者可采用如下策略:
            \begin{enumerate}
                \item[*] 任选等长但不同的 $m_1$ 和 $m_2$,并向 Oracle 询问得到 $t_1=F_k(F_k(m_1)\oplus\langle{|m_1|}\rangle)$ 和 $t_2=F_k(F_k(m_2)\oplus\langle{|m_2|}\rangle)$,构造 $m'=m_1\parallel\langle{|m_1|}\rangle\parallel{t_1}$,再向 Oracle 询问得到 $t'=F_k(F_k(F_k(F_k(m_1)\oplus\langle{|m_1|}\rangle)\oplus{t_1})\oplus\langle{|m'|}\rangle)$。
                \item[*] 输出 $m=m_2\parallel\langle{|m_2|}\rangle\parallel{t_2}$ 和 $t=F_k(F_k(F_k(F_k(m_2)\oplus\langle{|m_2|}\rangle)\oplus{t_2})\oplus\langle{|m|}\rangle)$。
            \end{enumerate}
            注意到 $m$ 和 $m'$ 等长且以下等式成立:
            \begin{enumerate}
                \item[*] $t'=F_k(F_k(F_k(F_k(m_1)\oplus\langle{|m_1|}\rangle)\oplus{t_1})\oplus\langle{|m'|}\rangle)=F_k(F_k(t_1\oplus{t_1})\oplus\langle{|m'|}\rangle)$,
                \item[*] $t=F_k(F_k(F_k(F_k(m_2)\oplus\langle{|m_2|}\rangle)\oplus{t_2})\oplus\langle{|m|}\rangle)=F_k(F_k(t_2\oplus{t_2})\oplus\langle{|m|}\rangle)$。
            \end{enumerate}
            因而,$\algo{Vrfy}_k(m,t)=1$,另一方面,显然 $m=m_2\parallel\langle{|m_2|}\rangle\parallel{t_2}$ 没有被查询过,即攻击者有不可忽略的概率攻击成功。
        \end{solution}

    \question (4.16题)[选做] Define a version of CBC-MAC for messages of length at most $\ell\cdot{2^n}$ as follows: given a message $m$, pad it with $0$s so that it has length exactly $\ell\cdot{2^n}$; apply basic CBC-MAC to the result. Is this secure?

        \begin{solution}
            \newline
            此方案不是安全的 MAC,攻击者可采用如下策略:构造 $m'=1$(一位二进制),并向 Oracle 询问得到 $t'=\algo{Mac_k}(1\parallel{0^{\ell\cdot2^n-1}})$,输出 $m=10$(两位二进制)和 $t=t'$。\\
            注意到 $t=\algo{Mac_k}(10\parallel{0^{\ell\cdot2^n-2}})=\algo{Mac_k}(1\parallel{0^{\ell\cdot2^n-1}})$,因而,$\algo{Vrfy}_k(m,t)=1$,另一方面,显然 $m=10$ 没有被查询过,即攻击者有不可忽略的概率攻击成功。
        \end{solution}

    \question (4.18题)[选做] Prove that the encoding for arbitrary-length messages described in Section 4.4.2 is prefix-free.

        \begin{solution}
            \newline
            对于不同的消息 $m_1$ 和 $m_2$ 考虑以下情况:
            \begin{enumerate}
                \item[*] $m_1$ 和 $m_2$ 的长度不同,注意到被填充的消息结构为 $\langle|m|\rangle\parallel0^t\parallel{m}$,两者的 $\langle|m|\rangle$ 部分将出现不同,因而填充结果也不同。
                \item[*] $m_1$ 和 $m_2$ 的长度相同,注意到被填充的消息结构为 $\langle|m|\rangle\parallel0^t\parallel{m}$,两者的 $\langle|m|\rangle$ 部分一致,而填充结果长度为 $n$ 的倍数,因而两者的 $0^t$ 部分也一致,由于 $m_1$ 和 $m_2$ 是不同的消息,两者的 $m$ 部分将出现不同,因而填充结果也不同。
            \end{enumerate}
            由上可知,不同的 $m_1$ 和 $m_2$ 填充的结果互不相同,即按照 4.4.2 中的方式对任意长度的消息进行填充是前缀自由的。
        \end{solution}

\end{questions}

\end{document}

%%% Local Variables:
%%% mode: latex
%%% TeX-master: t
%%% End:
