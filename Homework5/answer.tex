%% HOW TO USE THIS TEMPLATE:
%%
%% Ensure that you replace "YOUR NAME HERE" with your own name, in the
%% \studentname command below.  Also ensure that the "answers" option
%% appears within the square brackets of the \documentclass command,
%% otherwise latex will suppress your solutions when compiling.
%%
%% Type your solution to each problem part within
%% the \begin{solution} \end{solution} environment immediately
%% following it.  Use any of the macros or notation from the
%% header.tex that you need, or use your own (but try to stay
%% consistent with the notation used in the problem).
%%
%% If you have problems compiling this file, you may lack the
%% header.tex file (available on the course web page), or your system
%% may lack some LaTeX packages.  The "exam" package (required) is
%% available at:
%%
%% http://mirror.ctan.org/macros/latex/contrib/exam/exam.cls
%%
%% Other packages can be found at ctan.org, or you may just comment
%% them out (only the exam and ams* packages are absolutely required).

% the "answers" option causes the solutions to be printed
% \documentclass[11pt,addpoints]{exam}

%请使用xelatex编译
\documentclass[11pt,addpoints,answers]{exam}

% required macros -- get header.tex file from course web page
% !Mode:: "TeX:UTF-8"
% !TEX root=./hw1.tex
\usepackage{xeCJK}    % 写中文要用到
\usepackage{tikz}
\usepackage{amsmath,amsfonts,amssymb,amsthm}
\usepackage{fullpage}
\usepackage{times}
\usepackage{hyperref}
\usepackage{pdfsync}
\usepackage{microtype}
\usepackage{color}
\usepackage{cleveref}
\crefformat{footnote}{#2\footnotemark[#1]#3}
\definecolor{light-gray}{gray}{0.5}

\renewcommand{\tablename}{表}
\renewcommand{\figurename}{图}
\renewcommand{\proof}{证明}

% \theoremstyle{plain}
% \newtheorem{theorem}{定理~}
% \newtheorem{lemma}{引理~}
% \newtheorem{axiom}{公理~}
% \newtheorem{proposition}{命题~}
% \newtheorem{prop}{性质~}
% \newtheorem{corollary}{推论~}
% \newtheorem{conclusion}{结论~}
% \newtheorem{definition}{定义~}
% \newtheorem{conjecture}{猜想~}
% \newtheorem{example}{例~}
% \newtheorem{remark}{注~}
% \newtheorem{algorithm}{算法~}

%%% BLACKBOARD SYMBOLS

% \newcommand{\C}{\ensuremath{\mathbb{C}}}
\newcommand{\D}{\ensuremath{\mathbb{D}}}
\newcommand{\F}{\ensuremath{\mathbb{F}}}
% \newcommand{\G}{\ensuremath{\mathbb{G}}}
\newcommand{\J}{\ensuremath{\mathbb{J}}}
\newcommand{\N}{\ensuremath{\mathbb{N}}}
\newcommand{\Q}{\ensuremath{\mathbb{Q}}}
\newcommand{\R}{\ensuremath{\mathbb{R}}}
\newcommand{\T}{\ensuremath{\mathbb{T}}}
\newcommand{\Z}{\ensuremath{\mathbb{Z}}}
\newcommand{\QR}{\ensuremath{\mathbb{QR}}}

\newcommand{\Zt}{\ensuremath{\Z_t}}
\newcommand{\Zp}{\ensuremath{\Z_p}}
\newcommand{\Zq}{\ensuremath{\Z_q}}
\newcommand{\ZN}{\ensuremath{\Z_N}}
\newcommand{\Zps}{\ensuremath{\Z_p^*}}
\newcommand{\ZNs}{\ensuremath{\Z_N^*}}
\newcommand{\JN}{\ensuremath{\J_N}}
\newcommand{\QRN}{\ensuremath{\QR_{N}}}
\newcommand{\QRp}{\ensuremath{\QR_{p}}}

%%% THEOREM COMMANDS

% following are "theorem" style
\theoremstyle{plain}
\newtheorem{theorem}{定理}[section]
\newtheorem{lemma}[theorem]{引理}
\newtheorem{corollary}[theorem]{推论}
\newtheorem{proposition}[theorem]{命题}
\newtheorem{claim}[theorem]{声明}
\newtheorem{fact}[theorem]{事实}

% following are def style
\theoremstyle{definition}
\newtheorem{definition}[theorem]{定义}
\newtheorem{conjecture}[theorem]{推测}
\newtheorem{example}[theorem]{例}
\newtheorem{protocol}[theorem]{协议}

% following are remark style
\theoremstyle{remark}
\newtheorem{remark}[theorem]{注}
\newtheorem{note}[theorem]{注}
\newtheorem{exercise}[theorem]{练习}

% equation numbering style
\numberwithin{equation}{section}

%%% GENERAL COMPUTING

\newcommand{\bit}{\ensuremath{\set{0,1}}}
\newcommand{\pmone}{\ensuremath{\set{-1,1}}}

% asymptotics
\DeclareMathOperator{\poly}{poly}
\DeclareMathOperator{\polylog}{polylog}
\DeclareMathOperator{\negl}{negl}
\newcommand{\Otil}{\ensuremath{\tilde{O}}}

% probability/distribution stuff
\DeclareMathOperator*{\E}{E}
\DeclareMathOperator*{\Var}{Var}

% sets in calligraphic type
\newcommand{\calA}{\ensuremath{\mathcal{A}}}
\newcommand{\calD}{\ensuremath{\mathcal{D}}}
\newcommand{\calF}{\ensuremath{\mathcal{F}}}
\newcommand{\calH}{\ensuremath{\mathcal{H}}}
\newcommand{\calK}{\ensuremath{\mathcal{K}}}
\newcommand{\calM}{\ensuremath{\mathcal{M}}}
\newcommand{\calX}{\ensuremath{\mathcal{X}}}
\newcommand{\calY}{\ensuremath{\mathcal{Y}}}

% types of indistinguishability
\newcommand{\compind}{\ensuremath{\stackrel{c}{\approx}}}
\newcommand{\statind}{\ensuremath{\stackrel{s}{\approx}}}
\newcommand{\perfind}{\ensuremath{\equiv}}

% font for general-purpose algorithms
\newcommand{\algo}[1]{\ensuremath{\mathsf{#1}}}
% font for general-purpose computational problems
\newcommand{\problem}[1]{\ensuremath{\mathsf{#1}}}
% font for complexity classes
\newcommand{\class}[1]{\ensuremath{\mathsf{#1}}}

% complexity classes and languages
\renewcommand{\P}{\class{P}}
\newcommand{\BPP}{\class{BPP}}
\newcommand{\NP}{\class{NP}}
\newcommand{\coNP}{\class{coNP}}
\newcommand{\AM}{\class{AM}}
\newcommand{\coAM}{\class{coAM}}
\newcommand{\IP}{\class{IP}}

%%% "LEFT-RIGHT" PAIRS OF SYMBOLS

% inner product
\newcommand{\inner}[1]{\langle{#1}\rangle}
\newcommand{\innerfit}[1]{\left\langle{#1}\right\rangle}
% absolute value
\newcommand{\abs}[1]{\lvert{#1}\rvert}
\newcommand{\absfit}[1]{\left\lvert{#1}\right\rvert}
% a set
\newcommand{\set}[1]{\{{#1}\}}
\newcommand{\setfit}[1]{\left\{{#1}\right\}}
% parens
\newcommand{\parens}[1]{({#1})}
\newcommand{\parensfit}[1]{\left({#1}\right)}
% tuple = alias for parens
\newcommand{\tuple}[1]{\parens{#1}}
\newcommand{\tuplefit}[1]{\parensfit{#1}}
% square brackets
\newcommand{\bracks}[1]{[{#1}]}
\newcommand{\bracksfit}[1]{\left[{#1}\right]}
% rounding off
\newcommand{\round}[1]{\lfloor{#1}\rceil}
% floor function
\newcommand{\floor}[1]{\lfloor{#1}\rfloor}
% ceiling function
\newcommand{\ceil}[1]{\lceil{#1}\rceil}
% length of a string
\newcommand{\len}[1]{\lvert{#1}\rvert}
\newcommand{\lenfit}[1]{\left\lvert{#1}\right\rvert}
% length of some vector, element
\newcommand{\length}[1]{\lVert{#1}\rVert}
\newcommand{\lengthfit}[1]{\left\lVert{#1}\right\rVert}

%%% CRYPTO-RELATED NOTATION

% KEYS AND RELATED

\newcommand{\key}[1]{\ensuremath{#1}}

\newcommand{\pk}{\key{pk}}
\newcommand{\vk}{\key{vk}}
\newcommand{\sk}{\key{sk}}
\newcommand{\mpk}{\key{mpk}}
\newcommand{\msk}{\key{msk}}
\newcommand{\fk}{\key{fk}}
\newcommand{\id}{id}
\newcommand{\keyspace}{\ensuremath{\mathcal{K}}}
\newcommand{\msgspace}{\ensuremath{\mathcal{M}}}
\newcommand{\ctspace}{\ensuremath{\mathcal{C}}}
\newcommand{\tagspace}{\ensuremath{\mathcal{T}}}
\newcommand{\idspace}{\ensuremath{\mathcal{ID}}}
\newcommand{\concat}{\ensuremath{\|}}

% GAMES

% advantage
\newcommand{\advan}{\ensuremath{\mathbf{Adv}}}

% different attack models
\newcommand{\attack}[1]{\ensuremath{\text{#1}}}

\newcommand{\atk}{\attack{atk}}        % dummy attack
\newcommand{\indcpa}{\attack{ind-cpa}}
\newcommand{\indcca}{\attack{ind-cca}}
\newcommand{\anocpa}{\attack{ano-cpa}} % anonymous
\newcommand{\anocca}{\attack{ano-cca}}
\newcommand{\euacma}{\attack{eu-acma}} % forgery: adaptive chosen-message
\newcommand{\euscma}{\attack{eu-scma}} % forgery: static chosen-message
\newcommand{\suacma}{\attack{su-acma}} % strongly unforgeable

% ADVERSARIES

\newcommand{\attacker}[1]{\ensuremath{\mathcal{#1}}}

\newcommand{\Adv}{\attacker{A}}
\newcommand{\AdvA}{\attacker{A}}
\newcommand{\AdvB}{\attacker{B}}
\newcommand{\Dist}{\attacker{D}}
\newcommand{\Sim}{\attacker{S}}
\newcommand{\Ora}{\attacker{O}}
\newcommand{\Inv}{\attacker{I}}
\newcommand{\For}{\attacker{F}}

% CRYPTO SCHEMES

\newcommand{\scheme}[1]{\ensuremath{\text{#1}}}

% pseudorandom stuff
\newcommand{\prg}{\algo{PRG}}
\newcommand{\prf}{\algo{PRF}}
\newcommand{\prp}{\algo{PRP}}

% symmetric-key cryptosystem
\newcommand{\skc}{\scheme{SKC}}
\newcommand{\skcgen}{\algo{Gen}}
\newcommand{\skcenc}{\algo{Enc}}
\newcommand{\skcdec}{\algo{Dec}}

% public-key cryptosystem
\newcommand{\pkc}{\scheme{PKC}}
\newcommand{\pkcgen}{\algo{Gen}}
\newcommand{\pkcenc}{\algo{Enc}}
\newcommand{\pkcdec}{\algo{Dec}}

% digital signatures
\newcommand{\sig}{\scheme{SIG}}
\newcommand{\siggen}{\algo{Gen}}
\newcommand{\sigsign}{\algo{Sign}}
\newcommand{\sigver}{\algo{Ver}}

% message authentication code
\newcommand{\mac}{\scheme{MAC}}
\newcommand{\macgen}{\algo{Gen}}
\newcommand{\mactag}{\algo{Tag}}
\newcommand{\macver}{\algo{Ver}}

% key-encapsulation mechanism
\newcommand{\kem}{\scheme{KEM}}
\newcommand{\kemgen}{\algo{Gen}}
\newcommand{\kemenc}{\algo{Encaps}}
\newcommand{\kemdec}{\algo{Decaps}}

% identity-based encryption
\newcommand{\ibe}{\scheme{IBE}}
\newcommand{\ibesetup}{\algo{Setup}}
\newcommand{\ibeext}{\algo{Ext}}
\newcommand{\ibeenc}{\algo{Enc}}
\newcommand{\ibedec}{\algo{Dec}}

% hierarchical IBE (as key encapsulation)
\newcommand{\hibe}{\scheme{HIBE}}
\newcommand{\hibesetup}{\algo{Setup}}
\newcommand{\hibeext}{\algo{Extract}}
\newcommand{\hibeenc}{\algo{Encaps}}
\newcommand{\hibedec}{\algo{Decaps}}

% binary tree encryption (as key encapsulation)
\newcommand{\bte}{\scheme{BTE}}
\newcommand{\btesetup}{\algo{Setup}}
\newcommand{\bteext}{\algo{Extract}}
\newcommand{\bteenc}{\algo{Encaps}}
\newcommand{\btedec}{\algo{Decaps}}

% trapdoor functions
\newcommand{\tdf}{\scheme{TDF}}
\newcommand{\tdfgen}{\algo{Gen}}
\newcommand{\tdfeval}{\algo{Eval}}
\newcommand{\tdfinv}{\algo{Invert}}
\newcommand{\tdfver}{\algo{Ver}}

%%% PROTOCOLS

\newcommand{\out}{\text{out}}
\newcommand{\view}{\text{view}}

%%% COMMANDS FOR LECTURES/HOMEWORKS

\newcommand{\lecheader}{
\chead{
    \large \textbf{Lecture \lecturenum\\\lecturetopic}
}

\lhead{\small
    \textbf{\href{http://bhpan.buaa.edu.cn}{可证明安全技术}\\2022年秋季}
}

\rhead{\small
    \textbf{主讲: 张宗洋
}

\setlength{\headheight}{20pt}
\setlength{\headsep}{16pt}
}}

\newcommand{\hwheader}{
\chead{
    \Large \textbf{作业 \hwnum}
}

\lhead{\small
    \textbf{{可证明安全技术}\\2022年秋季}
}

\rhead{\small
    \textbf{主讲: 张宗洋\\姓名: \studentname}
}

\setlength{\headheight}{20pt}
\setlength{\headsep}{16pt}
\headrule
}

\newcommand{\examheader}{
\chead{
    \Large \textbf{测试 \\ \duedate}
}

\lhead{\small
    \textbf{\href{http://bhpan.buaa.edu.cn}{Introduction to modern cryptography}\\2022年秋季}}
    \rhead{\small \textbf{主讲: 张宗洋\\姓名: \studentname}}
    \setlength{\headheight}{20pt}
    \setlength{\headsep}{16pt}
    \headrule
}

\newcommand{\hint}[1]{\href{http://www.cims.nyu.edu/~regev/cgi-bin/hints/index.html?#1}{{\textcolor{light-gray}{I need a hint! (ID #1)}}}}

\newcommand{\hinttext}[2]{\href{http://www.cims.nyu.edu/~regev/cgi-bin/hints/index.html?#1}{{\textcolor{light-gray}{#2 (ID #1)}}}}

\newcommand{\hintcost}[2]{\href{http://www.cims.nyu.edu/~regev/cgi-bin/hints/index.html?#1}{{\textcolor{light-gray}{I need a hint for #2 points! (ID #1)}}}}

\newcommand{\moreinfo}[1]{\href{http://www.cims.nyu.edu/~regev/cgi-bin/hints/index.html?#1}{{\textcolor{light-gray}{I'm done solving and want to know more! (ID #1)}}}}


% VARIABLES

\newcommand{\hwnum}{5}
\newcommand{\studentname}{李田所-114514}

% END OF SUPPLIED VARIABLES

\hwheader                       % execute homework commands

\begin{document}

\pagestyle{head}                % put header on every page

% QUESTIONS START HERE.  PROVIDE SOLUTIONS WITHIN THE "solution"
% ENVIRONMENTS FOLLOWING EACH QUESTION.

\begin{questions}
    \question (12.2题) Show that for any CPA-secure public-key encryption scheme for single bit messages, the length of the ciphertext must be superlogarithmic in the security parameter.

        \begin{center}
            $\algo{Hint:}$ If not, the range of possible ciphertexts has polynomial size.
        \end{center}

        \begin{solution}
            \newline
            假设密文长度是安全参数的对数量级,对给定的公钥 $pk$ 和比特 $m$,可能的密文数量为安全参数的多项式量级,记可能的密文集合为 $C_{pk}(m)$,其大小至多为安全参数 $n$ 的多项式 $f(n)$。攻击者可采用以下策略:
            \begin{enumerate}
                \item[*] 对于输入 $pk$,输出明文 $m_0=0$ 和 $m_1=1$,得到密文 $c$。
                \item[*] 若 $c=c_0=\skcenc_{pk}(0)$,输出 $0$,若 $c=c_1=\skcenc_{pk}(1)$,输出 $1$,否则随机输出 $0$ 或 $1$。
            \end{enumerate}
            注意到存在 $c'\in{C}_{pk}(m)$ 满足 $\Pr[\skcenc_{pk}(m)=c']\geq\frac{1}{f(n)}$,可知:
            \begin{equation}
                \begin{aligned}
                    \Pr[\algo{PubK}^{eav}_{\Adv,\Pi}=1]=&\frac{1}{2}\Pr[b'=0|b=0]+\frac{1}{2}\Pr[b'=1|b=1]\\
                    =&\frac{1}{2}(\Pr[c=c_0]+\frac{1}{2}\Pr[c\neq{c}_0])+\frac{1}{2}(\Pr[c=c_1]+\frac{1}{2}\Pr[c\neq{c}_1])\\
                    =&\frac{1}{2}(\frac{1}{2}+\frac{1}{2}\Pr[c=c_0])+\frac{1}{2}(\frac{1}{2}+\frac{1}{2}\Pr[c=c_1])\\
                    =&\frac{1}{2}+\frac{1}{4}(\Pr[c=c_0]+\Pr[c=c_1])\\
                    \geq&\frac{1}{2}+\frac{1}{4}(\Pr[c=c'\wedge{c}_0=c']+\Pr[c=c'\wedge{c}_1=c'])\\
                    \geq&\frac{1}{2}+\frac{1}{4}(\frac{1}{f(n)^2}+\frac{1}{f(n)^2})=\frac{1}{2}+\frac{1}{2f(n)^2}
                \end{aligned}
            \end{equation}
            由于 $f(n)$ 为安全参数 $n$ 的多项式,$\frac{1}{2f(n)^2}$ 不可忽略,因而假设不成立,密文长度是安全参数的对数量级时,方案不具有 CPA 安全,密文长度应为安全参数的超对数量级。
        \end{solution}

    \question (12.3题) Say a public-key encryption scheme $(\skcgen,\skcenc,\skcdec)$ is $one$-$way$ if any $\algo{PPT}$ adversary $\AdvA$ has negligible probability of success in the following experiment:

        \begin{enumerate}
            \item[$\bullet$] $\skcgen\left(1^{n}\right)$ is run to obtain keys $(pk,sk)$.
            \item[$\bullet$] A uniform message $m$ in the message space is chosen, and a ciphertext $c\leftarrow\skcenc_{pk}(m)$ is computed.
            \item[$\bullet$] $\AdvA$ is given $pk$ and $c$, and outputs a message $m^{'}$. We say $\AdvA$ succeeds if $m^{'}=m$.
        \end{enumerate}

        \begin{enumerate}
            \item[(a)] Construct a CPA-secure KEM in the random-oracle model based on a one-way public-key encryption scheme with message space $\{0,1\}^n$.
            \item[(b)] Can a $deterministic$ public-key encryption scheme be one-way? If not, prove impossibility; if so, give a construction based on any of the assumptions introduced in this book.
        \end{enumerate}

        \begin{solution}
            \begin{enumerate}
                \item[(a)] 设 $\Pi=(\skcgen,\skcenc,\skcdec)$ 为一个单向的公钥加密方案,可构造如下 KEM:
                    \begin{enumerate}
                        \item[*] $\skcgen$:对输入 $1^n$,运行 $\skcgen$ 以获得公私钥 $(pk,sk)$,选定 $H: \set{0,1}^*\rightarrow\set{0,1}^{l(n)}$。
                        \item[*] $\hibeenc$:对输入的公钥 $pk$,选择 $r\in\set{0,1}^n$,输出密文 $\skcenc_{pk}(r)$ 和密钥 $H(r)$。
                        \item[*] $\hibedec$:对输入的私钥 $sk$ 和密文 $c$,输出密钥 $H(\skcdec_{sk}(c))$。
                    \end{enumerate}
                    下面说明上述构造满足 CPA 安全,设 Query 为 $\AdvA$ 向随机 Oracle 查询 $r$ 的情况,则:
                    \begin{equation}
                        \begin{aligned}
                            \Pr[\algo{KEM}^{cpa}_{\Adv,\Pi}=1]=&\Pr[\algo{KEM}^{cpa}_{\Adv,\Pi}=1\wedge\overline{Query}]+\Pr[\algo{KEM}^{cpa}_{\Adv,\Pi}=1\wedge{Query}]\\
                            \leq&\Pr[\algo{KEM}^{cpa}_{\Adv,\Pi}=1\wedge\overline{Query}]+\Pr[Query]
                        \end{aligned}
                    \end{equation}
                    若 $\Pr[\overline{Query}]=0$,则 $\Pr[\algo{KEM}^{cpa}_{\Adv,\Pi}=1\wedge\overline{Query}]=0$,否则有:
                    \begin{equation}
                        \begin{aligned}
                            \Pr[\algo{KEM}^{cpa}_{\Adv,\Pi}=1\wedge\overline{Query}]=&\Pr[\algo{KEM}^{cpa}_{\Adv,\Pi}=1|\overline{Query}]\cdot\Pr[\overline{Query}]\\
                            \leq&\Pr[\algo{KEM}^{cpa}_{\Adv,\Pi}=1|\overline{Query}]
                        \end{aligned}
                    \end{equation}
                    注意到在实验 $\algo{KEM}^{cpa}_{\Adv,\Pi}$ 中,$\AdvA$ 的输入为公钥 $pk$、密文 $\skcenc_{pk}(r)$ 和 $k=H(r)$ 或随机选择的 $k\in\set{0,1}^{l(n)}$,在不进行 Query 的情况下,由于 $H$ 为随机 Oracle,对攻击者而言 $k$ 是均匀分布的,攻击者无法根据 $k$ 进行区分;又因为密文 $\skcenc_{pk}(r)$ 采用单向的加密方案得到,其被攻破的概率可忽略,因而有:
                    \begin{equation}
                        \begin{aligned}
                            \Pr[\algo{KEM}^{cpa}_{\Adv,\Pi}=1|\overline{Query}]&\leq\frac{1}{2}+negl(n)\\
                            \Pr[\algo{KEM}^{cpa}_{\Adv,\Pi}=1]&\leq\frac{1}{2}+negl(n)+\Pr[Query]
                        \end{aligned}
                    \end{equation}
                    注意到 $\AdvA$ 为概率多项式时间的算法,其进行 Query 的次数存在多项式上限 $t(n)$,可基于针对 KEM 的攻击者 $\AdvA'$ 构造针对单向加密方案的攻击者 $\AdvA$:
                    \begin{enumerate}
                        \item[*] 对输入的公钥 $pk$ 和密钥 $c$,随机选择 $k\in\set{0,1}^{l(n)}$,随机选择 $b\in\set{0,1}$,$b=0$ 则设置 $\hat{k}=k$,否则随机选择 $\hat{k}\in\set{0,1}^{l(n)}$,使用 $pk,c,\hat{k}$ 启动算法 $\AdvA'$。
                        \item[*] 当 $\AdvA'$ 询问 $\widetilde{r}$ 时,如果 $\widetilde{r}$ 被询问过则返回上一次被询问时的返回值,如果 $\skcenc_{pk}(\widetilde{r})=c$ 则返回 $k$,否则随机选择 $k'\in\set{0,1}^{l(n)}$ 并返回。
                        \item[*] 当 $\AdvA'$ 运行结束时,从被询问过的 $\widetilde{r}$ 中随机返回一个结果。
                    \end{enumerate}
                    显然 $\AdvA$ 是概率多项式时间的算法,由于单向加密方案是安全的,其被攻破的概率可忽略且至少为 $\Pr[Query]/t(n)$,因而 $\Pr[Query]$ 可忽略,从而 $\Pr[\algo{KEM}^{cpa}_{\Adv,\Pi}=1]\leq\frac{1}{2}+negl(n)$,构造的 KEM 是 CPA 安全的。
                \item[(b)] 朴素的 RSA 加密算法即为一个满足确定性和单向性的构造,其确定性显然,下面假设其不具有单向性而存在概率多项式时间的攻击者 $\AdvA'$,则可构造基于 $\AdvA'$ 的,可解决 RSA 问题的方案 $\AdvA$:
                    \begin{enumerate}
                        \item[*] 对输入的 $(N,e,y)$,使用 $(N,e)$(即 $pk$)启动 $\AdvA'$。
                        \item[*] 对 $\AdvA'$ 输入 $y$,将 $\AdvA'$ 的输出 $m'$ 作为 $\AdvA$ 的输出。
                    \end{enumerate}
                由于 $\AdvA'$ 有不可忽略的概率攻击成功,攻击成功时 $m=m'=y^d\mod{N}$,因而 $m'^e\mod{N}=(y^d)^e\mod{N}=y^{de}\mod{N}=y\mod{N}$,$\AdvA$ 亦有不可忽略的概率攻击成功,这与 RSA 假设矛盾,故在 RSA 问题是困难问题的情况下,上述构造是单向的。
            \end{enumerate}
        \end{solution}

    \question (12.5题) Show that Claim 12.7 does not hold in the setting of CCA-security.

        \begin{solution}
            \newline
            攻击者可采取如下策略:
            \begin{enumerate}
                \item[*] 对于输入 $pk$,输出明文 $m_0=00$ 和 $m_1=11$,得到密文 $c=c_1\parallel{c}_2$。
                \item[*] 计算 $c_2'=\skcenc_{pk}(0)$,向解密 Oracle 查询 $c'=c_1\parallel{c}_2'$。
                \item[*] 记得到的明文 $m'=b\parallel0$,直接输出 $b$。
            \end{enumerate}
            显然,$m_0$ 和 $m_1$ 等长且 $c=c'=c_1\parallel{c}_2'$ 即 $c_2=\skcenc_{pk}(0)$ 的概率可忽略,因而攻击者有不可忽略的概率攻击成功,声明 12.7 中给出的方案 $\Pi'$ 不满足 CCA 安全。
        \end{solution}

    \question (12.6题) Consider the following public-key encryption scheme. The public key is $(G,q,g,h)$ and the private key is $x$, generated exactly as in the ElGamal encryption scheme. In order to encrypt a bit $b$, the sender does the following:

        \begin{enumerate}
            \item[(a)] If $b=0$ then choose a uniform $y\in{Z_q}$ and compute $c_1:=g^y$ and $c_2:=h^y$. The ciphertext is $\langle{c_1,c_2}\rangle$.
            \item[(b)] If $b=1$ then choose independent uniform $y,z\in{Z_q}$, compute $c_1:=g^y$ and $c_2:=g^z$, and set the ciphertext equal to $\langle{c_1,c_2}\rangle$.
        \end{enumerate}

        Show that it is possible to decrypt efficiently given knowledge of $x$. Prove that this encryption scheme is CPA-secure if the decisional Diffie–Hellman problem is hard relative to $\mathcal{G}$.

        \begin{solution}
            \begin{enumerate}
                \item[*] 可通过如下方式解密:对于 $\langle{c_1,c_2}\rangle$,计算 $c_1^x=(g^y)^x=(g^x)^y=h^y$。若 $c_2=h^y=c_1^x$ 则输出 $1$,否则输出 $0$(对随机选择的 $z$,$c_2=g^z=h^y=c_1^x$ 的概率可忽略)。
                \item[*] 假设存在可以攻破上述加密方案的攻击者 $\AdvA'$,可构造基于攻击者 $\AdvA'$ 的解决 DDH 问题的方案 $\AdvA$ 如下:
                    \begin{enumerate}
                        \item[*] 对输入的 $(G,q,g,h,c_1,c_2)$,使用 $(G,q,g,h)$ 启动 $\AdvA'$,假设 $\AdvA'$ 输出 $m_0=0$ 和 $m_1=1$($m_0=1$ 和 $m_1=0$ 的情况同理)。
                        \item[*] 对 $\AdvA'$ 输入 $\langle{c_1,c_2}\rangle$,将 $\AdvA'$ 的输出作为 $\AdvA$ 的输出。
                    \end{enumerate}
                对输入的 $(G,q,g,h,c_1,c_2)$,有 $h=g^x$ 和 $c_1=g^y$:
                    \begin{enumerate}
                        \item[*] 若 $c_2=g^z$($z$ 为随机选择),则有:
                            \begin{equation}
                                \begin{aligned}
                                    \Pr[D(G,q,g,g^x,g^y,g^z)]=\Pr[\algo{PubK}^{eav}_{\Adv,\Pi}=1|b=1]
                                \end{aligned}
                            \end{equation}
                        \item[*] 若 $c_2=h^y=g^{xy}$,则有:
                            \begin{equation}
                                \begin{aligned}
                                    \Pr[D(G,q,g,g^x,g^y,g^{xy})]=\Pr[\algo{PubK}^{eav}_{\Adv,\Pi}=0|b=0]
                                \end{aligned}
                            \end{equation}
                    \end{enumerate}
                由此可知:
                    \begin{equation}
                        \begin{aligned}
                            |\Pr&[D(G,q,g,g^x,g^y,g^z)]-\Pr[D(G,q,g,g^x,g^y,g^{xy})]|\\
                            &=|\Pr[\algo{PubK}^{eav}_{\Adv,\Pi}=1|b=1]-\Pr[\algo{PubK}^{eav}_{\Adv,\Pi}=0|b=0]|\\
                            &=|\Pr[\algo{PubK}^{eav}_{\Adv,\Pi}=1|b=1]-1+\Pr[\algo{PubK}^{eav}_{\Adv,\Pi}=1|b=0]|\\
                            &=|2\cdot\Pr[\algo{PubK}^{eav}_{\Adv,\Pi}=1]-1|=2\cdot|\Pr[\algo{PubK}^{eav}_{\Adv,\Pi}=1]-\frac{1}{2}|
                        \end{aligned}
                    \end{equation}
                由于攻击者 $\AdvA'$ 可以攻破上述加密方案,$\Pr[\algo{PubK}^{eav}_{\Adv,\Pi}=1]>\frac{1}{2}+negl(n)$,则 $2\cdot|\Pr[\algo{PubK}^{eav}_{\Adv,\Pi}=1]-\frac{1}{2}|>negl(n)$。\\
                因而 $|\Pr[D(G,q,g,g^x,g^y,g^z)]-\Pr[D(G,q,g,g^x,g^y,g^{xy})]|>negl(n)$,与 DDH 假设矛盾,故在 DDH 问题是困难问题的情况下,上述方案是 CPA 安全的。
            \end{enumerate}
        \end{solution}

    \question (12.8题) Consider the following protocol for two parties $A$ and $B$ to flip a fair coin (more complicated versions of this might be used for Internet gambling):

        \begin{enumerate}
            \item A trusted party $T$ publishes her public key $pk$; then $A$ choose a uniform bit $b_A$, encrypts it using $pk$, and announces the ciphertext $c_A$ to $B$ and $T$.
            \item Next, $B$ acts symmetrically and announces a ciphertext $c_B\neq{c_A}$; $T$ decrypts both $c_A$ and $c_B$, and the parties XOR the results to obtain the value of the coin.
        \end{enumerate}

        \begin{enumerate}
            \item[(a)] Argue that even if A is dishonest (but B is honest), the final value of the coin is uniformly distributed.
            \item[(b)] Assume the parties use El Gamal encryption (where the bit $b$ is encoded as the group element $g^b$ before being encrypted—note that efficient decrypt is still possible). Show how a dishonest $B$ can bias the coin to any value he likes.
            \item[(c)] Suggest what type of encryption scheme would be appropriate to use here. Can you define an appropriate notion of security and prove that your suggestion achieves this definition?
        \end{enumerate}

        \begin{solution}
            \begin{enumerate}
                \item[(a)] 注意到 $A$ 向 $B$ 和 $T$ 广播 $c_A$ 先于 $B$ 选择不同于 $c_A$ 的 $c_B$,攻击者 $A$ 在产生 $c_A$ 时无法获知关于 $B$ 的额外信息。由于 $B$ 是诚实的,其输出是随机的,因而无论 $A$ 是否诚实,最后的结果都是随机的。
                \item[(b)] 记 $c_A=c_1\parallel{c}_2$,$c_B=c_1'\parallel{c}_2'$ 攻击者 $B$ 可采用以下策略:
                    \begin{enumerate}
                        \item[*] 如果 $B$ 希望最终的结果为 $0$,可随机选择 $r$ 并输出 $c_1'=c_1\cdot{g}^r$ 和 $c_2'=c_2\cdot{h}^r$,此时可得:
                            \begin{equation}
                                \begin{aligned}
                                    b_B=&c_2'\cdot({c_1'^x})^{-1}\\
                                    =&c_2\cdot{h}^r\cdot(c_1\cdot{g}^r)^{-x}\\
                                    =&b_A\cdot{h}^y\cdot{h}^r\cdot(g^y\cdot{g}^r)^{-x}\\
                                    =&b_A\cdot{g}^{xy}\cdot{g}^{xr}\cdot{g}^{-xy}\cdot{g}^{-xr}=b_A
                                \end{aligned}
                            \end{equation}
                        因而总有 $b_A\oplus{b}_B=b_A\oplus{b}_A=0$。
                        \item[*] 如果 $B$ 希望最终的结果为 $1$,可随机选择 $r$ 并输出 $c_1'=c_1^{-1}\cdot{g}^r$ 和 $c_2'=c_2^{-1}\cdot{g}\cdot{h}^r$,此时可得:
                            \begin{equation}
                                \begin{aligned}
                                    b_B=&c_2'\cdot({c_1'^x})^{-1}\\
                                    =&c_2^{-1}\cdot{g}\cdot{h}^r\cdot(c_1^{-1}\cdot{g}^r)^{-x}\\
                                    =&(b_A\cdot{h}^y)^{-1}\cdot{g}\cdot{h}^r\cdot((g^y)^{-1}\cdot{g}^r)^{-x}\\
                                    =&b_A^{-1}\cdot{h}^{-y}\cdot{g}\cdot{h}^r\cdot(g^{-y}\cdot{g}^r)^{-x}\\
                                    =&b_A^{-1}\cdot{g}^{-xy}\cdot{g}\cdot{g}^{xr}\cdot{g}^{xy}\cdot{g}^{-xr}\\
                                    =&g\cdot{b}_A^{-1}
                                \end{aligned}
                            \end{equation}
                        因而总有 $b_A\oplus{b}_B=b_A\oplus(1-b_A)=1$。
                    \end{enumerate}
                \item[(c)] 注意到上一问中不安全的原因在于 $B$ 可以基于 $c_A$ 不经过加密算法构造 $c_B$,为解决此问题,考虑采用 CCA 安全的加密方案。\\
                先定义公平抛币方案如下:若一个抛币方案对于有概率多项式能力的攻击者,无论其处于 $A$ 的位置还是 $B$ 的位置,抛币结果为 $0$ 的几率 $p(n)$ 都不超过 $\frac{1}{2}+negl(n)$,则此方案为公平抛币方案。\\
                假设存在可以攻破采用了 CCA 安全的加密方案 $\Pi$ 的抛币方案的攻击者 $\AdvA'$,注意到第一问中已经说明,攻击者处于 $A$ 的位置时无法有效的攻击,攻击者一定处于 $B$ 的位置。下面构造攻击者 $\AdvA$,其可以基于 $\AdvA'$ 攻破 $\Pi$,$\AdvA$ 可采用如下策略:
                    \begin{enumerate}
                        \item[*] 对于输入 $pk$,输出明文 $m_0=0$ 和 $m_1=1$,得到密文 $c$ 作为 $c_A$。
                        \item[*] 对处于 $B$ 位置的攻击者 $\AdvA'$ 输入密文 $c_A$,得到其输出 $c_B$(不同于 $c_A$)。
                        \item[*] 向解密 Oracle 查询 $c_B$ 得到 $b_B$,若 $b_B=1$ 则输出 $1$,否则输出 $0$。
                    \end{enumerate}
                此时可得:
                    \begin{equation}
                        \begin{aligned}
                            \Pr[\algo{PubK}^{eav}_{\Adv,\Pi}=1]=&\frac{1}{2}\Pr[b'=0|b=0]+\frac{1}{2}\Pr[b'=1|b=1]\\
                            =&\frac{1}{2}\Pr[\text{抛币结果为} 0|b=0]+\frac{1}{2}\Pr[\text{抛币结果为} 0|b=1]\\
                            =&\Pr[\text{抛币结果为} 0]=p(n)
                        \end{aligned}
                    \end{equation}
                由于 $p(n)$ 超过 $\frac{1}{2}+negl(n)$,这与 CCA 安全要求的 $\Pr[\algo{PubK}^{eav}_{\Adv,\Pi}=1]\leq\frac{1}{2}+negl(n)$ 矛盾,因而假设不成立,采用了 CCA 安全的加密方案的抛币方案是公平抛币方案。
            \end{enumerate}
        \end{solution}

    \question (12.10题) In Section 12.4.4 we showed that El Gamal encryption is malleable, and specifically that given a ciphertext $\langle{c_1,c_2}\rangle$ that is the encryption of some unknown message $m$, it is possible to produce a ciphertext $\langle{c_1,c'_2}\rangle$ that is the encryption of $\alpha\cdot{m}$ (for known $\alpha$). A receiver who receives both these ciphertext might be suspicious since both ciphertext share the first component. Show that it is possible to generate $\langle{c'_1,c'_2}\rangle$, that is the encryption of $\alpha\cdot{m}$, with $c'_1\neq{c_1}$ and $c'_2\neq{c_2}$.

        \begin{solution}
            \newline
            可构造 $\langle{c'_1,c'_2}\rangle=\langle{c_1\cdot{g}^r,c_2\cdot{h}^r\cdot\alpha}\rangle$,此时有 $c'_1\neq{c_1}$ 和 $c'_2\neq{c_2}$,且:
            \begin{equation}
                \begin{aligned}
                    m'=&c_2'\cdot({c_1'^x})^{-1}\\
                    =&c_2\cdot{h}^r\cdot\alpha\cdot(c_1\cdot{g}^r)^{-x}\\
                    =&m\cdot{h}^y\cdot{h}^r\cdot\alpha\cdot(g^y\cdot{g}^r)^{-x}\\
                    =&m\cdot{g}^{xy}\cdot{g}^{xr}\cdot\alpha\cdot{g}^{-xy}\cdot{g}^{-xr}\\
                    =&m\cdot\alpha
                \end{aligned}
            \end{equation}
        \end{solution}

\end{questions}

\end{document}

%%% Local Variables:
%%% mode: latex
%%% TeX-master: t
%%% End:
